\paragraph{Cloud}
The \gls{cloud} is the term used for rented computing power and data storage in data centers owned by third parties.
This is in contrast to in-house or on-premise servers.
An advantage of running software in the \gls{cloud} is that maintenance of hardware is out sourced.
If a hard drive or processor breaks down, it is the cloud vendor's responsibility to fix, and to provide failover mechanisms so a customer is not impacted.


Another advantage is the ability to scale up or down instantly on demand.
If a on-premise server is overloaded, the organization has to purchase more servers and configure them.
Just the shipping of hardware alone takes more time than requesting more compute power from a \gls{cloud} provider.
The cloud providers usually have so large data centers that they never ``run out'', as long as a customer is able to pay for it.
Some of the best known cloud providers today are Amazon with Amazon Web Services, Google with Google Cloud, and Microsoft with Azure.

\paragraph{Gitpod}
Gitpod is a \gls{cloud} based \acrfull{IDE}.
It is provided as a service, or it can be self hosted.
The idea behind \gls{Gitpod} is that a developer does not need to install the tools on their own machine.
Instead, a machine is provisioned at a cloud provider, and any tools are installed there.
The developer interfaces with this machine through a web based \acrshort{IDE}.
For Gitpod, the default \acrshort{IDE} is \gls{Theia}.
The source code is downloaded from an online source code host, such as \gls{GitHub}, and into a workspace on the provisioned machine. %TODO: cite?
