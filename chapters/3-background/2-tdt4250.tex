\paragraph{Rationale}
Because the target audience of the software solution (tree editor) are students at \acrshort{NTNU}, it is helpful to know how they work with \acrlong{MDD}.
Their use cases are the ones being solved, meaning the solution must be made with this context in mind.

\paragraph{\Acrshort{MDD} at \acrshort{NTNU}}
To do \acrlong{MDD} effectively, tools should be used.
In the course ``\textit{\gls{TDT4250} Advanced Software Design}''\footnote{Course description is available at \href{https://www.ntnu.edu/studies/courses/TDT4250\#tab=omEmnet}{https://www.ntnu.edu/studies/courses/TDT4250\#tab=omEmnet}.} at \acrshort{NTNU}, the chosen tools are in the \acrfull{EMF}~\cite{hallvardtraettebergEMFTDT4250NTNU2017}.
This includes the modeling language \gls{Ecore}, visual editors in \gls{Eclipse}, model validation logic, the code generator named ``GenModel''\footnote{The code generator is actually named ``codegen'', but users only see the configuration model called ``GenModel''.} (generator model), and more.
\Acrshort{EMF} is a battle-tested technology also used in certain industries, and is well integrated with the \gls{Eclipse}.
The course \gls{TDT4250} also uses \gls{Eclipse} as a case study for other software design concepts, such as modularity (plugin architecture) and dynamic systems (OSGi), and custom \acrlongpl{DSL} which automatically work with \gls{Eclipse}.
\Acrshort{EMF} is relevant for most or all of those concepts.

\paragraph{Development methodology}
Students are taught a methodology or approach for how to do modeling.
They start by specifying a problem space, for example bookkeeping an organization of employees or the courses in \acrshort{NTNU}, and then abstract the problem into a model.
The initial model is externalized as \gls{Ecore} by using a tree editor in \gls{Eclipse}.


Then an \textit{model instance} is made, based on the model, and filled with example data from the domain.
This model instance is used to test and verify that the model is appropriate for the problem space.
Adjustments are made to the model to accommodate any problems with the model instance.


Then validations can be created for the model, by one or both of the following approaches: writing \acrfull{OCL} into model annotations, or marking the model element with an annotation and implementing it as java code.
\Acrshort{OCL} is a \acrlong{DSL} for navigating models and evaluating expressions, and the \gls{Eclipse} can detect annotations with \acrshort{OCL} and evaluate them against the \gls{Ecore} model.
The other option, writing java code, requires the student to first create a new \textit{genmodel} file from the model (by using a menu in \gls{Eclipse}), generating a java code project from the model, and then writing validation logic into the generated code.
For the java code to be picked up, \gls{Eclipse} can start a new instance which installs the generated code as a plugin~\cite{hallvardtraettebergConstraintsValidationTDT42502020}.


Next up, when the model is deemed sufficient, and the most important validations are in place, the student can try to create a user interface.
One of several choices here is to create an \textit{\gls{Eclipse} plugin}.
\Acrshort{EMF} provides code generation for utilities used to integrate the model into an editor for \gls{Eclipse}.
The student uses the genmodel to create these, and tweaks the code if wanted.
Then everything is installed into \gls{Eclipse} by launching a new \gls{Eclipse} instance with the code installed as a plugin.


Lastly, the user interface can be tested.
The student creates a new model instance file, enters some example data from the domain, and runs validation logic.

\paragraph{Lecture materials}
The steps mentioned in the methodology above are available online in \cite{hallvardtraettebergEMFStepbystepTDT42502017,hallvardtraettebergConstraintsValidationTDT42502020,hallvardtraettebergEditingEcoreModel2017,hallvardtraettebergGenmodelTDT4250NTNU2017}.
This is an advantage, because they can by used used in this master's thesis as a basis for creating evaluations and acceptance criteria.
