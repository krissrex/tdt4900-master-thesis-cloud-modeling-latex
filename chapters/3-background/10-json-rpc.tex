
\Gls{JSON-RPC} is a stateless and lightweight protocol for doing \acrlongpl{RPC} (\acrshort{RPC}).
It works over any transport mechanism that can send and receive text.
The data in \gls{JSON-RPC} is sent as \gls{JSON}, an object structure serialization format originally from javascript~\cite{json-rpcworkinggroupJSONRPCSpecification2010}.\\

\acrshort{RPC} is a technique to start a procedure on a remote server, as the name suggests.
A procedure is synonymous with a function or method in programming.
In \gls{JSON-RPC}, they are called by specifying the name and the parameters in a \textit{Request object}.
A request object must have the properties \texttt{jsonrpc} and \texttt{method}, and \texttt{id} if it is not a notification.
It may have \texttt{params}.
Notifications, as explained for \acrshort{LSP}, do not need a response.
The \textit{Response object} must have the properties \texttt{jsonrpc}, \texttt{id}, and either \texttt{result} or \texttt{error}.
The result is the return value for the called procedure.
The error is an \textit{Error object}, with \texttt{code}, \texttt{message} and \texttt{data}~\cite{json-rpcworkinggroupJSONRPCSpecification2010}.\\

The request parameters in \texttt{params} are either a list of positional parameters, or an object with named parameters.
If there are no parameters, they can be omitted~\cite{json-rpcworkinggroupJSONRPCSpecification2010}.
An example of several \gls{JSON-RPC} messages are shown in \cref{lst:json-rpc-example}.
The arrows indicate the direction.

\begin{lstlisting}[caption={JSON-RPC examples copied from \cite{json-rpcworkinggroupJSONRPCSpecification2010}.}, label={lst:json-rpc-example}]
Syntax:
--> data sent to Server
<-- data sent to Client

RPC call with positional parameters:
--> {"jsonrpc": "2.0", "method": "subtract", "params": [42, 23], "id": 1}
<-- {"jsonrpc": "2.0", "result": 19, "id": 1}

RPC call with named parameters:
--> {"jsonrpc": "2.0", "method": "subtract", 
     "params": {"subtrahend": 23, "minuend": 42}, "id": 3}
<-- {"jsonrpc": "2.0", "result": 19, "id": 3}

Notifications:
--> {"jsonrpc": "2.0", "method": "update", "params": [1,2,3,4,5]}
--> {"jsonrpc": "2.0", "method": "foobar"}
\end{lstlisting}
