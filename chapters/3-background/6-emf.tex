
\paragraph{Rationale}
The \acrfull{EMF} is the \acrlong{MDD} framework used in \gls{TDT4250}.
The tree editor will modify \gls{Ecore} models, so it helps to understand the concepts and names used in the \gls{Ecore} metamodel.
It is also useful to know the different tools and components in \acrshort{EMF}, because the tree editor intends to reuse as much of them as possible internally, to save development effort.

\paragraph{\acrlong{EMF}}
The \acrfull{EMF} is a part of the Eclipse Modeling project from the Eclipse Foundation.
It is a framework and code generation facility that lets developers define models.
The models can be java code, \gls{XMI} or \gls{UML}, and the other two can be generated~\cite[p.~14]{edmerksEMFEclipseModeling2009}.
This framework may be chosen as the tools for doing \acrlong{MDD} (see \cref{sec:conceptual-modeling}).
In EMF, the models are expressed with the \gls{Ecore} modeling language.
This modeling language is similar to \gls{UML} Class Diagrams, in terms of the concepts and what it can express~\cite[p.~16]{edmerksEMFEclipseModeling2009}.
The real world data that could fit inside a specific model is called a \textit{model instance}.

The framework was made to take use of the editing capabilities and utility of the \gls{Eclipse}~\cite{edmerksEMFEclipseModeling2009}.
This means that there is much tooling and integration for \acrshort{EMF} with \gls{Eclipse}.
For example, EMF can generate a plugin to edit model instances in \gls{Eclipse}.

%TODO
% eclipse editor plugins, language, codegen, ocl, serialization format

\paragraph{\Gls{Ecore} metamodel}
The modeling language in \acrshort{EMF} is \gls{Ecore}.
A \textit{metamodel} is the model of a model.
This means that Ecore is the metamodel for all models expressed using \acrshort{Ecore}.
Ecore is itself modeled in Ecore, so it is its own metamodel.


\paragraph{Model concepts}
The main concepts used in \gls{Ecore} to model, are \texttt{EClass}, \texttt{EAttribute}, \texttt{EReference} and \texttt{EDataType}%
\footnote{The name Ecore comes from EMF Core, and the `E' prefix for \texttt{EClass} etc.\ come from Ecore.}.
These are distinct objects with names, properties and inheritance, like in object oriented programming.
As for the metamodel, \texttt{EClass}, \texttt{EAttribute} and \texttt{EReference} are all extending \texttt{ENamedElement}, which defines their \texttt{name} property~\cite{edmerksEMFEclipseModeling2009}.

When modeling, \texttt{EClass} is used to create java classes.
The \texttt{EAttribute} and \texttt{EReference} are used to model class properties, like member variables.
An \texttt{EAttribute} defines a property, such as e.g. \textit{age} or \textit{address}, while \texttt{EReference} defines a reference/association to another \texttt{EClass}, e.g. \textit{parent} or \textit{order}.
The \texttt{EAttribute} has a attribute type, the \texttt{EDataType}, which can be e.g. \texttt{EInt} or \texttt{EString}~\cite{edmerksEMFEclipseModeling2009}.

Java class methods are modeled with another concept, the \texttt{EOperation}.
Lastly, everything in the model lives inside an \texttt{EPackage}, which represents a java package (or other kind of code module).
There are more concepts in \gls{Ecore}, but many are only used internally as part of the metamodel, to represent \gls{Ecore} itself.


\paragraph{\Acrshort{XMI} serialization}
When an \gls{Ecore} model is written as a text file, it needs \textit{serialization}.
The official format for serializing Ecore is \acrfull{XMI}.
This format is based on \acrfull{XML}.
Model instances can also be serialized as \acrshort{XMI}.
It is also possible to serialize \gls{Ecore} to other formats, like \gls{JSON}, using third party tools.
% serialization, standardized EMOF, default for ecore, XML,

\paragraph{\acrshort{EMF} java \acrshort{API}}
The java code generated by \acrshort{EMF} will by default extend a set of java classes defined by \acrshort{EMF}.
Instead of a generated \texttt{EClass} extending \texttt{java.lang.Object}, it extends \texttt{EObject}.
And instead of using an \texttt{ArrayList}, a collection in \gls{Ecore} will use a \texttt{EList}.
When creating a new instance, the class constructor is not used, but a Factory instance on the generated \texttt{EPackage} for the model.


All of these framework java-classes are the \acrshort{EMF} java \gls{API}.
They provide much of the power, flexibility, reflection and meta-modeling capabilities of \acrshort{EMF} in java.
For example, a program can work with a \acrshort{EMF} model without knowing the code beforehand, by using the reflection \acrshort{API} to retrieve names and properties of a model object.


The \acrshort{API} also provides utilities for working with the model.
There are \acrshortpl{API} for listing the children of an \texttt{EObject}, getting a human representation of it, and for modifying and observing state changes.
Another important \acrshort{API} is the \texttt{ResourceSet} and \texttt{Resource}, used to read and save models to serialized \acrshort{XMI} files.

\paragraph{GenModel code generation}
Code generation is an important part of \acrshort{EMF}.
The generator can be configured with its own generator model, nicknamed the \textit{GenModel}.
This model holds options for how the code will be named, what templates should write the code, if the code can use the \acrshort{EMF} \acrshortpl{API}, and more.


The generator can also produce more than just a java representation of the model.
A test suite can be generated, with an empty test skeleton for the generated code.
It can also generate utilities for creating model editors, in what is called the \textit{.edit} java package.
The name ``.edit'' is appended to the original package name.
This has \textit{ItemProvider} classes which helps an editor to find the human representations, properties, child objects, and to notify on changes.


Another utility is related to the \gls{Eclipse}, which is the \textit{.editor} java package.
This holds key classes for integrating with \gls{Eclipse}, making it a custom editor.
For example, custom actions, project wizards, eclipse plugin logic is part of this.


\paragraph{Custom code}
The generated code must usually be modified by a developer.
This can be to fill in the implementation of a \texttt{EOperation}, or tweak some behavior.
The generated code has a \texttt{@Generated} java annotation, which the developer changes to prevent the code generator from overwriting the method body.
