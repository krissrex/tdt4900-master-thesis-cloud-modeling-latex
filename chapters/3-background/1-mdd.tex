\paragraph{Rationale}
\Acrfull{MDD} is the approach to software development which this thesis aims to support.
Therefore, and understanding of \acrshort{MDD} is beneficial, in order to see how an editor should work.

\paragraph{Modeling and abstraction}
The core of \acrshort{MDD} is the model.
The model is a human created construct, formed through humans working together to discuss and refine a problem domain until they reach a consensus of what abstractions help them solve the relevant problems~\cite[p.~154]{brambillaModeldrivenSoftwareEngineering2012}.
Humans perceive the world (and problem domain) as many different phenomena, and conceptual modeling is the act of trying to describe these at some level of abstraction~\cite[p.~1,408]{krogstieModelbasedDevelopmentEvolution2012}.
The model is assumed to resemble the phenomena and work the same way, and yet be simpler than the real world~\cite[p.~414]{krogstieModelbasedDevelopmentEvolution2012}.
Abstraction means to find something common in different observations of a phenomena, and \textit{generalize} their features, \textit{classify} coherent clusters of objects and \textit{aggregate} concepts into more complex ones~\cite[p.~1]{brambillaModeldrivenSoftwareEngineering2012}.
The model will never describe every aspect of the world perfectly, but can \textit{reduce} the world down to relevant aspects, and easily \textit{map} between model elements and real world phenomena~\cite[p.~1-2]{brambillaModeldrivenSoftwareEngineering2012}.


\paragraph{Modeling languages}
In order to describe the model, a \textit{language} is used.
To realize the benefits of \acrshort{MDD}, a \textit{formal language} is used.
The language can be textual or graphical, or both, and imposes a formally defined syntax on the modeler~\cite[p.~13]{brambillaModeldrivenSoftwareEngineering2012}.

\paragraph{Modeling tools}
The advantage of using a formal language is that it can be parsed and understood by software tools, as well as humans.
The tools can validate the model according to the syntax, and to specific rules for the domain.
Tools can also generate code, or execute the model itself.
The model can be transformed into other models, or text or graphics~\cite[p.~8]{brambillaModeldrivenSoftwareEngineering2012}.

\paragraph{\acrlong{MDD}}
The central idea of \acrlong{MDD} is that the model is the source of truth that \textit{drives} the rest of the engineering and development~\cite[p.~9]{brambillaModeldrivenSoftwareEngineering2012}.
There is not a separate model for analysis and for design, but a single one for both~\cite[p.~49]{evansDomaindrivenDesignTackling2004}.
The software code becomes an expression of the model itself, and changes to the code often happen as the result of changes to the model~\cite[p.~49]{evansDomaindrivenDesignTackling2004}.
Because the model and the software are so directly related, the \acrshort{MDD} approach is heavily reliant on tools to automate the tasks of validation and code generation.
The formal language may also sacrifice some of its human readability in order to be understood by tools~\cite[p.~232]{krogstieModelbasedDevelopmentEvolution2012}.
To solve this, one can use other tools that interpret, transform or present models in other ways~\cite[p.~233]{krogstieModelbasedDevelopmentEvolution2012}.
This increases the reliance on tools for \acrshort{MDD} even more, including visual editors.

