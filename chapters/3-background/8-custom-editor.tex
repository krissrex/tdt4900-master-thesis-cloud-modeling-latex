A \gls{VSCode} extension is allowed to use a set of \acrlongpl{API} provided by \gls{VSCode}.
One such \acrshort{API} is the \textit{Custom Editor API}.
This allows an extension developer to create \textbf{custom editors other than text editors}.
This could be diagrams, pictures, graphs, or \textbf{trees}, for example.
The developer has the full freedom of a web browser, as they are given their own isolated frame.
Normally, an extension cannot modify the user interface outside of the provided \acrshortpl{API}.
This is in contrast to inside the provided \texttt{WebView}, where the developer has to \textit{create and manage} the entire user interface.
In addition to a user facing \texttt{WebView}, the developer must create their own document model.
By default, \gls{VSCode} uses a document model for text documents, with selections, edits, versions and more.
The \texttt{CustomDocument} only has a uri pointing to the file.
Another central part is the \texttt{CustomEditorProvider}, with a few methods to fill in, like opening, undoing and saving a document.

