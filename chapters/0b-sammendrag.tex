\chapter*{Sammendrag}

Programvareutvikling har en tilnærming som kalles Model-Dreven Utvikling (MDD).
Dette undervises til studenter i høyere utdanning.
Tilnærmingen er avhengig av verktøy, og et slikt verktøy er Eclipse Modeling Framework (EMF).
Selv om EMF kan brukes for å lære studenter om MDD, er det upopulært på grunn av sin tilknytning til Eclipse Integrated Development Environment (IDE), som gjør at studenter stritter i mot å lære MDD.
Skybaserte alternativer til Eclipse IDE eksisterer, som Gitpod med VSCode, og de har nyttige egenskaper for en utdanningsorganisasjon.
Verktøyene i EMF finnes derimot ikke for disse alternativene.
Denne masteroppgaven prøver å legge til rette for å støtte EMF i de skybaserte alternativene.

Fremgangsmåten i masteroppgaven er basert på Design Science Research, hvor et design blir lagd og en programvare blir utviklet.
Designet drar inspirasjon fra Language Server Protocol (LSP) og Graphical Language Server Platform (GLSP), protokoller for tekst- og diagramredigering.
Disse protokollene brukes allerede i VSCode.

Resultatet er en utvidelse for VSCode for redigering av tre-strukturer.
EMF-modeller kan redigeres som trær.
Denne utvidelsen består av tre komponenter: et generisk brukergrensesnitt for tre-redigering, en utvidelse for VSCode, og en EMF-spesifikk tjener (server).
Utvidelsen og serveren snakker med en nylig designet protokoll: \acrfull{TLSP}.

Den resulterende programvaren kan bygges på videre, for å bruke EMF-modellering i skyen.
TLSP-protokollen og programvarearkitekturen kan brukes av også andre verktøy som trenger tre-redigering, og som sikter på å støtte flere IDE-er.
En utbredt bruk av TLSP i IDE-er vil gjøre at migrering av tre-redigeringsverktøy til andre IDE-er blir forenklet.
Uavhengig av dette, så gir designet en gjenbrukbar server for EMF, som kan forenkle migreringen av EMF til andre IDE-er.