
The current design (\textit{models} and \textit{constructs}) achieves the basic goals of modeling in a \gls{cloud} based \acrshort{IDE}.
Applying the architecture of a general editor and domain specific server seems to fit the problem of tree structure editors.

The developed artifact (\textit{instantiation}) is \textbf{not} ready to be used by students of \gls{TDT4250}, as much of the design is not implemented.
Some important features are not yet supported by the design, mainly related to code generation and validation.
They are not proven to be infeasible, but are unexplored. \\

Another unexplored area is the special adaptation of editors to GenModel and Ecore themselves.
The GenModel presented in \gls{Eclipse} has many options that are not present in the actual file itself.
This may mean that there are default options presented in the \gls{Eclipse}, and only saved to the genmodel file if changed.
The feasibility of a customized editor is unexplored. \\

An important observation is that creating a Tree Editor frontend from scratch requires many ``basic'' features to be manually made.
Keyboard shortcuts and right-click context menus are not present by default.
For example, undoing the text entered in a property form field must also be supported explicitly by the design; the \acrshort{IDE} does not provide it for free.
The amount of work may be substantial, to reach a high level of quality and useability.\\

When designing the protocol, two different paradigms of protocol designs were encountered.
One is the \acrshort{LSP} approach, where many methods and capabilities are explicit in the protocol.
The other is the \acrshort{GLSP} approach, where the protocol has approximately 3 methods in total, and all the logic is encapsulated in a \texttt{ActionMessage}.
The server will perform a \texttt{handler} lookup for the message, and thus move much of the ``surface area'' of the protocol to internal code and handlers.
Essentially, the \acrshort{GLSP} approach is wrapping the \gls{JSON-RPC} data inside another layer of \gls{JSON-RPC} data.
One observed advantage of this is the forwarding of \gls{JSON-RPC} calls: the \acrshort{GLSP} approach only needs to forward a \texttt{ActionMessage} and it is done.
The \acrshort{LSP} has to ``unpack and re-call'' the forwarded message.
One potential cause for the difference is that forwarding is much more present in \acrshort{GLSP}: it also uses a Custom Editor frontend.
Thus messages are relayed from the frontend via the extension and to the server.
In \acrshort{LSP}, the editor is natively supported by the \acrshort{IDE}.

The design of this thesis' artifact chose the \acrshort{LSP} approach, of a large and explicit protocol.
Without evidence, it is assumed that this approach is easier for \textit{other} developers when they want to implement a server for a domain other than \acrshort{EMF}.