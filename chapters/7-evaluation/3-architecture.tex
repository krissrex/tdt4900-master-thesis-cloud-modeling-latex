
The software architecture has been designed to keep \acrshort{EMF} out of the tree editor frontend and \acrshort{TLSP} protocol.
The server has been designed to function independently of the \acrshort{IDE}, when it comes to \acrshort{EMF} editing.
The server also reuses software from \acrshort{EMF} to avoid reimplementation.\\

\subsection{Reusable Components for Related Migrations}

\paragraph{Frontend}
The frontend instantiation can be further developed to become a high quality component.
This should provide enough utility and value that other \gls{VSCode} extension developers use it when they need to implement tree editors.\\
Currently, the tree editor frontend is not good enough for this.
It is missing a context menu (right click), keyboard navigation, drag-and-drop of nodes, navigation from the \gls{VSCode} problems menu and log\footnote{Text editors can navigate to a specific file and line when the user clicks a specific URI. These are used to link to stack trace source locations, warnings and problems.}.
The current styling is visually unpleasing, and the user experience is poor.
None of these are unsolvable problems, but do require substantial effort.
That is where the value of a single, reuseable and standard tree editor lies: avoiding the effort when it has already been done.

\paragraph{Protocol}
The protocol is designed independent of \acrshort{EMF}, on the same ideas of the editor frontend.
When the protocol is specified enough, so it is near complete and highly functional, different \acrshortpl{IDE} can start to support it natively.
The benefits of \acrshort{LSP} lies in this native \acrshort{IDE} support: the text editors are already made and high quality; the language server implementer only needs to create one component.
Currently, the protocol is underdeveloped.
The specified methods have matching functionality in the editor, but do not go further in ``fear'' of specifying invalid or unusable methods.\\

The protocol does not have an official specification document or website.
The \acrshort{LSP} has this, which serves as the authoritative source for the specification, along with Microsoft's implementation for \gls{VSCode}.
A protocol should be specified in clear, unambiguous language, and have a clear indication of who the authors and maintainers of the specification is.
A protocol should also be versioned and have a changelog.
The current design only specifies the protocol in source code.

\paragraph{Component distribution}
The current open source project is not very reuseable.
While the components have a design and architecture to support reusability, they are not published to artifact repositories like maven central and npm.
This is crucial for reusability in other extension developers' projects, as copying code or downloading compiled binaries is deemed ``dirty'', wrong, and likely to become outdated.

\subsection{Components for Migrating EMF to Other IDEs}

\paragraph{Reusable server}
The server is developed independent of \gls{VSCode}.
The only external dependencies are the Java 11 Runtime and a client for the \acrlong{TLSP}.
The server should contain all the logic related to \acrshort{EMF}.
The current instantiation has some knowledge inside the \gls{VSCode} extension, like the file types and \gls{VSCode} Command names.
This is because they must be known before extension activation.
A future and more complete first party integration of \acrshort{TLSP} into \acrshort{VSCode} could move this knowledge to the protocol instead.
Regardless, it is a small amount of \acrshort{EMF} knowledge, compared to what the server holds.
Transition to other \acrshort{IDE} will be easy, if they support \acrshort{TLSP} and can run the server component.

\paragraph{EMF code reuse}
The server is successfully able to reuse parts of the \acrshort{EMF} framework.
The \acrshort{EMF} runtime \acrshort{API}, the \gls{Ecore} metamodel, the \texttt{ReflectiveItemProvider} and the EMF.Cloud Model Server are all used.
A server should reuse other components as well, like the code generator, \gls{OCL} compiler/interpreter, validation framework and so on.
These are not reused in the instantiation yet, but because of time constraints.
They are not re-implemented either, which is the ``worst case scenario''.\\
