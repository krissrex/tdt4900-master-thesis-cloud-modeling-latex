
\subsection{Project Evaluation}

The software project created as part of this thesis has been evaluated according to \cref{sec:method-open-source}.
The results of the evaluation are shown in \cref{tab:open-source-evaluation}.
Some results are clarified further below, using the test requirement's ID as the paragraph name.

\paragraph{OS8} The project has a Changelog.md file for the \gls{VSCode} extension, but not for the other components or the project as a whole.

\paragraph{OS10} There is no official plan. But the project could potentially be promoted using EclipseCon 2021, or through email to related stakeholders such as EclipseSource members (like Dr. Jonas Helming).

\paragraph{OS11} The author is the committed person, but the duration of commitment cannot be guaranteed to be long enough.

\paragraph{OS18} Not all the code is commented, but the (subjectively) required parts are.

% Please add the following required packages to your document preamble:
% \usepackage{booktabs}
% \usepackage[table,xcdraw]{xcolor}
% If you use beamer only pass "xcolor=table" option, i.e. \documentclass[xcolor=table]{beamer}
% \usepackage{longtable}
% Note: It may be necessary to compile the document several times to get a multi-page table to line up properly
\begin{longtable}{@{}lp{9cm}l@{}}
\toprule
\multicolumn{1}{c}{\textbf{ID}} &
  \multicolumn{1}{c}{\textbf{Requirement}} &
  \multicolumn{1}{c}{\textbf{\begin{tabular}[c]{@{}c@{}}Present\\ {[}No/Yes{]}\end{tabular}}} \\* \midrule
\endfirsthead
%
\multicolumn{3}{c}%
{{\bfseries Table \thetable\ continued from previous page}} \\
\toprule
\multicolumn{1}{c}{\textbf{ID}} &
  \multicolumn{1}{c}{\textbf{Requirement}} &
  \multicolumn{1}{c}{\textbf{\begin{tabular}[c]{@{}c@{}}Present\\ {[}No/Yes{]}\end{tabular}}} \\* \midrule
\endhead
%
\bottomrule
\endfoot
%
\endlastfoot
%
\rowcolor[HTML]{EFEFEF} 
OS1                         & Open source license                                                              & Yes  \\
OS2                         & Readme                                                                           & Yes  \\
\rowcolor[HTML]{EFEFEF} 
OS3                         & Contributing guidelines                                                          & No   \\
OS4                         & Code of conduct                                                                  & No   \\
\rowcolor[HTML]{EFEFEF} 
OS5                         & \cellcolor[HTML]{EFEFEF}LICENSE file                                             & No   \\
OS6                         & NOTICE file                                                                      & No   \\
\rowcolor[HTML]{EFEFEF} 
\cellcolor[HTML]{EFEFEF}OS7 & Easy to remember project name                                                    & Yes  \\
OS8                         & Changelog file                                                                   & No*  \\
\rowcolor[HTML]{EFEFEF} 
OS9                         & No sensitive material in the commit history (passwords etc)                      & Yes  \\
OS10                        & You have a marketing plan for announcing and promoting the project               & No*  \\
\rowcolor[HTML]{EFEFEF} 
OS11                        & A person is committed to managing community interactions (issues, pull requests) & Yes* \\
OS12                        & Consistent code conventions and clear function/method/variable names             & Yes  \\
\rowcolor[HTML]{EFEFEF} 
OS13                        & Wiki or documentation website                                                    & No   \\
OS14                        & GitHub Issue templates for bugs and feature requests                             & No   \\
\rowcolor[HTML]{EFEFEF} 
OS15                        & GitHub Pull Request templates                                                    & No   \\
OS16                        & Semantic release versioning (using \texttt{major.minor.patch}, like \texttt{1.4.1})                & Yes  \\
\rowcolor[HTML]{EFEFEF} 
OS17 & Code dependencies (and transitive dependencies) do not use GPL or Sun BCLA & Yes \\
OS18                        & Commented and documented code, explaining intentions and edge cases              & Yes* \\* \bottomrule
\caption{Evaluation of the project as \gls{open source}.
The requirements are sourced from \textcite{mikelinksvayerStartingOpenSource2020}, \textcite{dannyguoMakeREADME2020}, \textcite{beatonThirdPartyContent2017} and \textcite{waynebeatonEclipseProjectHandbook2020}.
The results are in the ``Present'' column.
A result with an asterisk (*) is explained further in the text.}
\label{tab:open-source-evaluation}\\
\end{longtable}

\subsection{Readme Evaluation}

The root level ``Readme'' file is very important.
It is the face of the project to the world.
Many of the \gls{open source} evaluation elements are related to the Readme.
Therefore, they have been condensed to a separate evaluation.
Note that this project has multiple modules, and therefore multiple Readme files.
This evaluation only looks at the top level, project wide Readme.
The results of the evaluation are shown in \cref{tab:readme-evaluation}.
Some results are clarified further below, using the test requirement's ID as the paragraph name.

\paragraph{OSR3} The Readme does not have the description, but the \gls{GitHub} project has a description shown in the side of the webpage of the project.

\paragraph{OSR5} The Readme does not have a project background story, but the \gls{GitHub} project has a description explaining this is a result of a masters thesis.

% Please add the following required packages to your document preamble:
% \usepackage{booktabs}
% \usepackage[table,xcdraw]{xcolor}
% If you use beamer only pass "xcolor=table" option, i.e. \documentclass[xcolor=table]{beamer}
% \usepackage{longtable}
% Note: It may be necessary to compile the document several times to get a multi-page table to line up properly
\begin{longtable}{@{}lll@{}}
\toprule
\multicolumn{1}{c}{\textbf{ID}} &
  \multicolumn{1}{c}{\textbf{Requirement}} &
  \multicolumn{1}{c}{\textbf{\begin{tabular}[c]{@{}c@{}}Present\\ {[}No/Yes{]}\end{tabular}}} \\* \midrule
\endfirsthead
%
\multicolumn{3}{c}%
{{\bfseries Table \thetable\ continued from previous page}} \\
\toprule
\multicolumn{1}{c}{\textbf{ID}} &
  \multicolumn{1}{c}{\textbf{Requirement}} &
  \multicolumn{1}{c}{\textbf{\begin{tabular}[c]{@{}c@{}}Present\\ {[}No/Yes{]}\end{tabular}}} \\* \midrule
\endhead
%
\bottomrule
\endfoot
%
\endlastfoot
%
\rowcolor[HTML]{EFEFEF} 
OSR1                         & The project has a name.                           & Yes \\
OSR2 &
  \begin{tabular}[c]{@{}l@{}}Badges or icons (e.g. CI build status, npm version,\\ open-vsix store page, vscode store page)\end{tabular} &
  No \\
\rowcolor[HTML]{EFEFEF} 
OSR3                         & Description of what the project does.             & No* \\
OSR4                         & List of suported features.                        & No  \\
\rowcolor[HTML]{EFEFEF} 
OSR5                         & \cellcolor[HTML]{EFEFEF}Project background story. & No* \\
OSR6                         & How can a user use this project?                  & No  \\
\rowcolor[HTML]{EFEFEF} 
\cellcolor[HTML]{EFEFEF}OSR7 & Where can a user get more help?                   & No  \\
OSR8                         & Is the project ready for use?                     & No  \\
\rowcolor[HTML]{EFEFEF} 
OSR9                         & Feature roadmap                                   & No  \\
OSR10                        & Contributing description.                         & No  \\
\rowcolor[HTML]{EFEFEF} 
OSR11                        & Getting started with contributing.                & No  \\
OSR12                        & Authors and acknowledgement                       & No  \\
\rowcolor[HTML]{EFEFEF} 
OSR13                        & Project license                                   & No  \\
OS14 &
  \begin{tabular}[c]{@{}l@{}}Project status (e.g. active, lost interest,\\ discontinued, deprecated, looking for new owner, migrated)\end{tabular} &
  No \\* \bottomrule
\caption[Open Source Evaluation of the Readme File]{Evaluation of an \gls{open source} project's Readme.
The requirements are sourced from \textcite{mikelinksvayerStartingOpenSource2020}, \textcite{dannyguoMakeREADME2020}, \textcite{beatonThirdPartyContent2017} and \textcite{waynebeatonEclipseProjectHandbook2020}.
The results are in the ``Present'' column.
A result with an asterisk (*) is explained further in the text.
}
\label{tab:readme-evaluation}\\
\end{longtable}
