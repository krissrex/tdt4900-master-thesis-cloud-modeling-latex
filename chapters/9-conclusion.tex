\chapter{Conclusion}\label{chap:conclusion}

This thesis started by finding a way to enable modeling using \acrlong{EMF} in the \gls{cloud}, to solve Objective 1.
A pre-project suggested creating a tree editor for \gls{Gitpod}, by creating a \gls{VSCode} extension.
This thesis then designed an extension which used a software architecture and protocol that drew inspiration from the \acrlong{LSP} and \acrlong{GLSP} designs.
The thesis also presented a way to implement a tree editor with this architecture, and a method which extracts functional requirements from the existing \acrshort{EMF} tree editors in \gls{Eclipse} through use cases.
The architecture is a three component system: a generic tree editor frontend, a \gls{VSCode} extension, and an \acrshort{EMF}-specific server.
This architecture also entailed using a protocol, dubbed the \acrfull{TLSP}, between the \gls{VSCode} extension and the server.
This design was implemented up to the point it could successfully render \acrshort{EMF} models in \gls{Gitpod} with \gls{VSCode}.\\

The successful design and implementation of this architecture means it can be the basis of a new sibling protocol to \acrshort{LSP} and \acrshort{GLSP}: the \acrshort{TLSP}.
This can be used by other tools that undergo a similar migration to the \gls{cloud}. 
This design also created encapsulated and reusable components.
It can also ease future migration of \acrshort{EMF} to another \acrlong{IDE}, because the \acrshort{EMF} logic is contained in a reusable server, independent of \gls{VSCode}.
This means it solves Objective 3: An architecture to enable future related IDE migrations.\\

Lastly, this thesis produced an \gls{open source} project for the editor and \acrshort{TLSP}.
However, open sourcing software requires more effort than what is currently done to attract contributions.
Gaining traction online for \gls{open source} is similar to product marketing.
The amount of effort required to create a viable solution for \gls{cloud} based tree editing in \gls{VSCode} is substantial.
