\chapter{Conclusion}\label{chap:conclusion}

This thesis started by finding a way to enable modeling using \acrlong{EMF} in the \gls{cloud}, to solve Objective 1.
A pre-project suggested creating a tree editor for \gls{Gitpod}, by creating a \gls{VSCode} extension.
This thesis then designed an extension which used a software architecture and protocol that drew inspiration from the \acrlong{LSP} and \acrlong{GLSP} designs.
The thesis also presented a way to implement a tree editor with this architecture, and a method which extracts functional requirements from the existing \acrshort{EMF} tree editors in \gls{Eclipse} through use cases.
The architecture is a three component system: a generic tree editor frontend, a \gls{VSCode} extension, and an \acrshort{EMF}-specific server.
This architecture also entailed using a protocol, dubbed the \acrfull{TLSP}, between the \gls{VSCode} extension and the server.
This design was implemented up to the point it could successfully render \acrshort{EMF} models in \gls{Gitpod} with \gls{VSCode}.\\

The successful design and implementation of this architecture means it can be the basis of a new sibling protocol to \acrshort{LSP} and \acrshort{GLSP}: the \acrshort{TLSP}.
This can be used by other tools that undergo a similar migration to the \gls{cloud}. 
This design also created encapsulated and reusable components.
It can also ease future migration of \acrshort{EMF} to another \acrlong{IDE}, because the \acrshort{EMF} logic is contained in a reusable server, independent of \gls{VSCode}.
This means it solves Objective 3: An architecture to enable future related IDE migrations.\\

Lastly, this thesis produced an \gls{open source} project for the editor and \acrshort{TLSP}.
However, open sourcing software requires more effort than what is currently done to attract contributions.
Gaining traction online for \gls{open source} is similar to product marketing.
The amount of effort required to create a viable solution for \gls{cloud} based tree editing in \gls{VSCode} is substantial.
External contributions may be required for completing the implementation.



\section{Future Work}

This thesis has uncovered some new potential areas of research.

\paragraph{Theia or VSCode as a deployment platform}
The GenModel and code generator can target \gls{Eclipse} as a deployment platform.
The model gets a plugin generated, and an editor in \gls{Eclipse} for model instances.
A similar approach can be interesting for deploying a tree editor for model instances, but using \gls{Theia} or \gls{VSCode} instead of \gls{Eclipse}.
These could potentially reuse the \acrshort{TLSP} as well.
Doing this would increase the value of \acrshort{EMF}, and further illustrate the values of \acrlong{MDD} and code generation.

\paragraph{Collaborative modeling}
Modeling is a collaborative task.
With the move to cloud, and with the increased amount of work from home\footnote{Due to covid-19.}, collaboration can move online as well.
This is already normal for things like Google Docs, and Jetbrains just added ``Code With Me''.
Obeo is also developing this for their cloud based Sirius modeling tool.
When all the editing is done through a protocol like \acrshort{TLSP}, the data could be redirected to multiple clients, meaning multiple students' computers.

\paragraph{Completing the EMF editor}
A lot of the remaining work is routine design, not research.
But some remaining parts may be more challenging, such as getting the GenModel working and specializing the editor to properly show a \texttt{.genmodel} file like in \gls{Eclipse}.
There are also challenges to loading the generated code back into the server, for validations and custom \texttt{ItemProviders}.
Completing this will make the editor more useful, and also move it closer to a solution suited for industry use, beyond just education.


