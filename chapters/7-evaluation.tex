\chapter{Evaluation}\label{chap:evaluation}

\section{Use Case Evaluation of Tree Editor Extension}

% Please add the following required packages to your document preamble:
% \usepackage{booktabs}
% \usepackage[table,xcdraw]{xcolor}
% If you use beamer only pass "xcolor=table" option, i.e. \documentclass[xcolor=table]{beamer}
% \usepackage{longtable}
% Note: It may be necessary to compile the document several times to get a multi-page table to line up properly
\begin{longtable}{@{}lp{3cm}ll@{}}
\caption{Use Case-evaluation of the Tree Editor Extension design.}
\label{tab:use-case-evaluation}\\
\toprule
\multicolumn{1}{c}{\textbf{ID}} &
  \multicolumn{1}{c}{\textbf{Use Case}} &
  \multicolumn{1}{c}{\textbf{\begin{tabular}[c]{@{}c@{}}Supported?\\ {[}No/Yes/YES/Unknown{]}\end{tabular}}} &
  \multicolumn{1}{c}{\textbf{\begin{tabular}[c]{@{}c@{}}Requires \\ Eclipse IDE\end{tabular}}} \\* \midrule
\endfirsthead
%
\multicolumn{4}{c}%
{{\bfseries Table \thetable\ continued from previous page}} \\
\toprule
\multicolumn{1}{c}{\textbf{ID}} &
  \multicolumn{1}{c}{\textbf{Use Case}} &
  \multicolumn{1}{c}{\textbf{\begin{tabular}[c]{@{}c@{}}Supported?\\ {[}No/Yes/YES/Unknown{]}\end{tabular}}} &
  \multicolumn{1}{c}{\textbf{\begin{tabular}[c]{@{}c@{}}Requires \\ Eclipse IDE\end{tabular}}} \\* \midrule
\endhead
%
\bottomrule
\endfoot
%
\endlastfoot
%
\rowcolor[HTML]{EFEFEF} 
1  & Create new .ecore model file                                 & Yes     & No  \\
2  & View Ecore model by opening the .ecore file                  & YES     & No  \\
\rowcolor[HTML]{EFEFEF} 
3  & Create an EPackage, EClass and EAttributes and EReferences   & Yes     & No  \\
4  & Change the properties of the package, class and attributes   & Yes     & No  \\
\rowcolor[HTML]{EFEFEF} 
5  & Create a new dynamic instance file from an EClass            & Yes     & No  \\
6  & Enter dynamic instance data                                  & Yes     & No  \\
\rowcolor[HTML]{EFEFEF} 
7  & Change the .ecore model by adding a EAttribute to the EClass & Yes     & No  \\
8 &
  \begin{tabular}[p{3cm}]{@{}l@{}}Open the dynamic instance, confirm if it is marked as invalid\\ because the new attribute is not filled in.\end{tabular} &
  Unknown &
  Yes \\
\rowcolor[HTML]{EFEFEF} 
9 &
  \begin{tabular}[c]{@{}l@{}}Open the .ecore model file, and add a new validation\\ to the EClass as an EAnnotation. Use the java validation kind,\\ not OCL.\end{tabular} &
  Yes &
  No \\
10 & Create a .genmodel file based on the .ecore file.            & Yes     & No  \\
\rowcolor[HTML]{EFEFEF} 
11 & Generate java project with the model code                    & Unknown & Yes \\
12 & Write a validation in the java code                          & -       & No  \\
\rowcolor[HTML]{EFEFEF} 
13 & Load the model code into the IDE, to use the validation      & No      & Yes \\
14 & Edit the model and run the custom validation                 & Yes     & Yes \\
\rowcolor[HTML]{EFEFEF} 
15 & Generate a user interface or editor plugin                   & No      & Yes \\* \bottomrule
\end{longtable}

% TODO

\section{Test of Value for Students from Tree Editor}

* Feature test based on Confluence tasks.
  * Result: missing many features still. Not usable in education.

\section{Test of Value for Extension Developers from Protocol}
* Protocol-to-usecase parity. (Protocol-to-emfCommands parity?)
  * Result: JSON-RPC part is missing most features still. Not stable and recommended for adoption. The extension needs more features in order to grow the protocol. The data structure is valid; it can represent different tree structures from domains like Ecore, json and file systems.

\section{Acceptance Test of Tree Editor Extension}

\subsection{Runtime Environment Test}
* Environment end-to-end test in gitpod.
  * Result: the extension can install and render Ecore diagrams.

\subsection{Flexibility Test}
* Flexibility test to change the label or which properties are shown as children.
  * Result: failed. No such feature yet. No major blocker except development time.

\subsection{Functionality Test}
* Editor functionality test.
  * Result: Can view multiple trees and select nodes. Can show/hide children. Does not show correct icon. Missing drag-and-drop, editing, keyboard shortcuts, right click, keyboard navigation, validation.

\section{Test of Viability of Open Source Project}
* Project viability test.
  * Result: has a good kernel. Missing issues and roadmap on github. Clean code, but many TODO/FIXME that are unresolved. 
  