\chapter{Evaluation}\label{chap:evaluation}

% TODO

\section{Test of Value for Students from Tree Editor}

* Feature test based on Confluence tasks.
  * Result: missing many features still. Not usable in education.

\section{Test of Value for Extension Developers from Protocol}
* Protocol-to-usecase parity. (Protocol-to-emfCommands parity?)
  * Result: JSON-RPC part is missing most features still. Not stable and recommended for adoption. The extension needs more features in order to grow the protocol. The data structure is valid; it can represent different tree structures from domains like Ecore, json and file systems.

\section{Acceptance Test of Tree Editor Extension}

\subsection{Runtime Environment Test}
* Environment end-to-end test in gitpod.
  * Result: the extension can install and render Ecore diagrams.

\subsection{Flexibility Test}
* Flexibility test to change the label or which properties are shown as children.
  * Result: failed. No such feature yet. No major blocker except development time.

\subsection{Functionality Test}
* Editor functionality test.
  * Result: Can view multiple trees and select nodes. Can show/hide children. Does not show correct icon. Missing drag-and-drop, editing, keyboard shortcuts, right click, keyboard navigation, validation.

\subsection{Test of Viability of Open Source Project}
* Project viability test.
  * Result: has a good kernel. Missing issues and roadmap on github. Clean code, but many TODO/FIXME that are unresolved. 
  