\chapter{Discussion}\label{chap:discussion}

The discussion will look at the implications of the results and their evaluations in a bigger picture.
Arguments will be presented based on these results, eventually leading to the conclusions in the next chapter.

\section{VSCode as an EMF Tree Editor in the Cloud}
When moving the \acrlong{EMF} to the cloud, \gls{VSCode} is a suitable \acrshort{IDE} for this.
It can run as a \gls{cloud} \acrshort{IDE} in \gls{Gitpod}.
\gls{VSCode} also provides enough mechanisms to build a tree editor, without unreasonable amounts of effort.

\paragraph{VSCode is a better choice than Theia}
While \gls{Theia} is a product of the Eclipse ecosystem's efforts, it remains a deployment target rather than a standalone \acrshort{IDE}.
Theia is replaced in \gls{Gitpod} with \gls{VSCode} as the default \acrshort{IDE}.
The developers struggle to keep feature parity with \gls{VSCode}, as they have to catch up whenever \gls{VSCode} moves forward, while at the same time managing their other tools and components\footnote{The main developers behind \gls{Theia} are the same as those behind \gls{Gitpod} and \gls{Che}, meaning they are spreading their efforts thinly.}.
The Custom Editor \acrshort{API} for example, came much later to \gls{Theia}\footnote{The API was added in March 2021 to Theia. VSCode released it to the public in March 2020.}.
Even if \gls{Theia} have more tools from the Eclipse ecosystem for deploying \acrshort{EMF} models, these mostly rely on the Theia Extension mechanism, incompatible with \gls{Gitpod}\footnote{Deploying a Theia Extension requires compilation of Theia, and replacing the entire IDE in Gitpod. Remember that a Theia \textit{Plugin} is equivalent to the VSCode extension concept. A Theia Extension is different and not compatible at all with VSCode.}.

\paragraph{VSCode is extensible enough}
\gls{VSCode} does not provide tools for creating tree editors.
However, the extension \acrshort{API} provided by \gls{VSCode} has enough features to build a tree editor.
The different file extensions can be associated with a customized document editor.
There are \acrshortpl{API} to perform Commands, such as requesting the extension to create a new file, or some other arbitrary action.
Developed extensions can be distributed and installed without extra work, bureaucracy or fees/costs.
This makes \gls{VSCode} a good candidate for extending with \acrshort{EMF} editor capabilities.

\paragraph{VSCode extensions can show a tree editor}
The Custom Editor \acrshort{API} from \gls{VSCode} allows an extension to freely render an editor using web technologies like HTML, javascript and CSS.
This is enough flexibility to create a custom tree editor to show hierarchical tree structures with labels and icons, and a property sheet.
The tree editors in \gls{Eclipse} can be re-implemented on a functional level as a Custom Editor in \gls{VSCode}.

\paragraph{VSCode extensions can run compiled programs}
The \acrlong{EMF} relies heavily on java as it exists now.
Because \gls{VSCode} can run compiled programs such as executable \texttt{.jar} files, the existing \acrshort{EMF} code can be ran under \gls{VSCode}.
However, this code needs to communicate across processes to integrate it into a \gls{VSCode} extension.
\gls{VSCode} itself has no way of reaching inside the process, but can use standard mechanisms like streams (\textit{stdin/stdout}), sockets, and HTTP requests.
This makes it possible to avoid re-implementing all of \acrshort{EMF} for a javascript runtime.

\section{Reuse of EMF java code}

Using a server component in the tree editor allows implementing it with java.
This in turn opens for reuse of the existing \acrshort{EMF} java runtime \acrshort{API}, related \acrshort{EMF} libraries and the components used in \gls{Eclipse}.

\paragraph{Java server that reuses EMF}
Because \gls{VSCode} extensions can run compiled programs, and not only javascript extension code, java can be used.
This in turn opens for using the official \acrshort{EMF} runtime \acrshort{API}, and the official \gls{Ecore} metamodel implementation.
These allow reflective access to any \acrshort{EMF} models.
It also saves effort, because it reuses existing code.
A contributing \gls{open source} developer already familiar with \acrshort{EMF} will also find it intuitive to contribute on such a server component.
It also removes the risk of using third party \gls{EMF} runtime \acrshort{API} implementations, like CrossEcore, which are not as battle tested as the java one\footnote{This solution tried using CrossEcore-generated TypeScript in the editor frontend, but scrapped it because the underlying library and code generator had bugs.}.\\

Additionally, it lets the server implementation use any \acrshort{EMF} tools already developed for java, like solutions for storage, change management, change-transactions and so on.

\paragraph{The EMF.Cloud Model Server can be embeded in the server}
Instead of adding another process, another communication step, and another components to manage, the EMF.Cloud Model Server can be used from java.
The server is itself made with java, and has an extensible design.
The design implementation in this thesis demonstrated that this kind of use is possible.
Instead of having a \gls{REST} \acrshort{API} library control the EMF.Cloud Model Server, the internally used components are extracted and put behind the server's \acrshort{TLSP} implementation.
These components implement several editing-related features, like scanning for al model files, loading \acrshort{EMF} \texttt{Resource}s and other useful features\footnote{It provides \texttt{EditingDomain}s, and processes \acrshort{EMF} \texttt{Command}s It also provides UI Schemas for models, to use with JSON-Forms.}.


\section{Creating a Tree Editor for VSCode Requires Substantial Effort}

Even if \gls{VSCode} is \textit{able} to support a tree editor for \acrshort{EMF}, the editor itself has to be created.
With the current situation, this is a lot of effort.

\paragraph{Many functional requirements}
No good source can provide all the functional requirements needed for a \acrshort{EMF} tree editor, in a clear and concise manner.
Existing literature does not provide it, so it must be extracted.
And because this thesis did not have it as an objective, little effort was spent on formally capturing and describing the requirements.
The thesis did however attempt to find requirements as an input to the design process, by simulating use cases.
The discovery is that the amount of requirements is high, and it is hard to separate between necessities and nice-to-haves.
A proper effort to create \textit{the} tree editor component for all tree editing uses, other than \acrshort{EMF}, needs to do the requirements engineering more formally.
Especially when the functionality needs to be replicated for other \acrshortpl{IDE} if they are to support \acrshort{TLSP}.

\paragraph{Little comes out of the box}
Regarding the development of a tree editor in \gls{VSCode}, even the most basic features must be implemented.
For example, a text input field in a browser will normally let a user right click and select ``undo'', or press \texttt{ctrl Z} to undo.
In a \gls{VSCode} custom editor, this is not there.
No context menu shows up at all when right clicking.
The reasoning might be that the Custom Editor is a blank slate, for any kind of editor.
But as a developer, one would prefer if an existing framework could provide most of the common and trivial functionality.
Luckily, the custom editor is still able to use any javascript libraries and frameworks that target a web browser.
If any existing components can provide a context menu or framework for building editors, they can be used.
A catch is that they have to conform to the stateless and remote-controlled nature of the Custom Editor's WebView.

\paragraph{High usability demands high effort}
Because so little comes out of the box, many rudimentary features that are assumed and expected by a user will have to be implemented.
The existing \gls{Eclipse} and its tools for \acrshort{EMF} were disliked partially for problems with the tools.
A tree editor in the cloud might not be judged on any ``lighter'' terms by a student, and can be perceived equally bad or worse.
Installation issues and problems with the cloud based editor need to be minimized, to provide a good user experience.
And the functionality may have to match the existing high standards of quality that are seen elsewhere on the web and \acrshort{IDE} space.
Neglecting this may just create another inferior editor, which works against the purpose of increasing \acrshort{MDD} and \acrshort{EMF} adoption.

\paragraph{Open sourcing requires investment before payoff}
Publishing a project online is not ``open sourcing'' it.
And developers will not flock to the project as free manpower.
As evident in the evaluation, a lot of work has to be done on the Readme and project documentation.
\Gls{open source} projects almost have to be ``sold'' to developers, like commercial products.
The Readme must clearly convey the purpose, the novelty and usefulness, the functionality and guide the contribution setup for the \gls{open source} project.
The project also needs a good, memorable name.
Then it has to be marketed to interested developers, so they can discover it.
And once contributors come, the project needs a person to manage it, processing any new issues and feature requests, reviewing incoming contribution code and merging pull requests.
Only after all this, comes the direct payoffs for the project itself.\\

However, there is another benefit of being \gls{open source}.
Even if no one wants to contribute, the project can be used as inspiration, code can be copied or ``scavenged for parts'' into other projects.
This project did this itself, by taking the Base Protocol jsonrpc implementation from the \gls{open source} LSP4J and VSCode LSP projects.


\section{Designing a Standardized Tree Language Server Protocol}

When \acrshort{EMF} first is in the process of moving from \gls{Eclipse} to \gls{VSCode}, choices to make another future migration easier may pay off.
Standardizing on a protocol for all \gls{cloud} based tree editors, and also isolating all \acrshort{EMF} editing to a single server component can support this.

\paragraph{Tree editors can use a standardized protocol}
The design work in this thesis has shown that \acrshort{EMF} can be mapped to a generic tree structure.
This tree structure is free of all \acrshort{EMF} concepts, and thus other domains with tree structures\footnote{For example HTML, XML, JSON and file systems.} could map to the same generic tree structure, and immediately fit into the developed tree editor frontend.
This tree editor frontend is also demonstrated to be instructed about domain specifics (\acrshort{EMF}) using a generic tree editor protocol: \acrshort{TLSP}.
Standardizing on such a protocol can give the same benefits seen in \acrshort{LSP} in solving the $m\times{}n$ language--\acrshort{IDE} problem, and more out-of-the-box support for a language when moving to a new \acrshort{IDE} which supports \acrshort{LSP}.
Eventually, with good adoption, such a protocol could reduce migrations of \acrshort{EMF} to different \acrshortpl{IDE} to become trivial.

\paragraph{EMF may need to migrate IDE again}
New \acrshortpl{IDE} show up.
Trends change.
\gls{VSCode} itself launched in 2015.
\gls{Eclipse} launched in 2001, while \acrshort{EMF}\footnote{The earliest version the author could find in the Eclipse EMF release page was v1.0.2 in 2003.} launched in 2003.
While it has been 14 years between \gls{Eclipse} and \gls{VSCode}, it was only 2 years between \gls{VSCode} and \gls{Theia}.
When \acrshortpl{IDE} become more component-based and reuseable, the available \acrshortpl{IDE} in the market may increase\footnote{A parallel is seen in web browsers, where Google's open source Chromium also is the basis for Microsoft Edge, Vivaldi, Brave, and others.}.
\gls{Gitpod} itself changed from \gls{Theia} to \gls{VSCode}.
If a solution had implemented \acrshort{EMF} editing in \gls{Gitpod} a year earlier, tightly coupled to \gls{Theia}, it would already need to transition to \gls{VSCode}.
There are also non-cloud \acrshortpl{IDE} where \acrshort{EMF} could be integrated.
For example, IntelliJ is a popular \acrshortpl{IDE} for Java, without any \acrshortpl{EMF} support.
A premise for this migration, however, is that the other \acrshortpl{IDE} can implement \acrshort{TLSP} and a tree editor frontend.

\paragraph{LSP has a suitable Base Protocol}
While the \acrlong{LSP} is designed for text editing, its Base Protocol is generic and reusable.
This Base Protocol can be specialized to support a tree editor protocol.
A big win for this protocol is that it is proven to work for both \acrshort{LSP} and \acrshort{GLSP}, for text and diagrams respectively, \textbf{and} it has readily available components to create new software.
The extension in this design reused the \texttt{vscode-jsonrpc} component, directly from the \acrshort{LSP} project.
The server used the java version of this, from the \texttt{Eclipse LSP4J} project.
Getting started with the Base Protocol has essentially no upfront development cost.
If the design is similarly component-based and \acrshort{LSP} independent for the other programming languages supported by \acrshort{LSP}, the Base Protocol may already be available as a dependency/library in other programming languages too.\\

Another advantage of using the Base Protocol, is that it is bi-directional.
This allows the server to initiate a request to the client.
When working with \acrshort{EMF}, the model can be changed by the server or other editors (like a diagram editor).
This bi-directional protocol then allows the server to notify any editor of changes.

\paragraph{Tree editors can use TLSP}
The standardized protocol can be created on the Base Protocol and become analogous to \acrshort{LSP} for trees.
The design created in this thesis demonstrates that this is feasible.
Thus, tree editors can standardize around the \acrfull{TLSP}.
The current protocol design may need changes, but it should be sufficient as a starting point.

\paragraph{TLSP design creates reusable software}
Regardless of a wider adoption of \acrshort{TLSP} or not, its design when applied to \acrshort{EMF} tree editing will create more reusable software.
The editor frontend can be reused by an unrelated tree editing project.
The \acrshort{EMF} \acrshort{TLSP} server can be reused as an alternative to EMF.Cloud Model Server.
If the server contains the core of \acrshort{EMF} editing, the external interface (\acrshort{TLSP}) could be replaced if not needed\footnote{Just like the implemented server did with EMF.Cloud Model Server's \gls{REST} \acrshort{API}.}.




\section{Limitations}

Research is not always perfect, and creative design is hard to do in an objective manner.
This section addresses some limitations, shortcomings and uncertainties of this thesis.

\paragraph{An unufinished instantiation leaves risk in the design}
Because development takes longer time than what is allocated for a master's thesis, a substantial amount the protocol design is left on a theoretical design stage.
Many challenges and constrains appear first when the design is implemented, imposed by the surrounding frameworks, \acrshortpl{API} and tools used to create the software.
Although the author is confident in this thesis' use case evaluation, the results cannot be guaranteed with the utmost certainty before they are all implemented.
Regardless, the overarching design and protocol should still be valid and contribute new knowledge.
Most obstacles encountered from here on, are assumed to be possible to work around with minor changes to the protocol.

\paragraph{Agile development causes a protocol to emerge slowly}
Doing an Agile approach means creating software that works.
It also means postponing work, instead of having a lot of unfinished work in progress.
This is good for delivering working software to customers and users, but not for developing a full and complete protocol specification.
While an agile approach can always back up the validity and usefulness of protocol elements with a working implementation, it may leave unidentified risks for later, contrary to how research aims to tackle risk early.
It can be argued, however, that Design Science Research uses the build-evaluate cycle exactly to produce correct knowledge, instead of drafts and unproven hypotheses.

\paragraph{Lacking contact with stakeholders and Eclipse ecosystem}
The developers of tools like Theia Tree Editor, ecore-glsp, JSON-Forms etc. may have deeper insights and knowledge than the author.
They may be able to suggest better, more pragmatic and effective design solutions.
They can also verify or disprove, or at least indicate, the need for generic tree editors and protocols for \acrshort{EMF}.
The author did not actively communicate with the Eclipse ecosystem, outside of two private chats during EclipseCon 2020.
There has also not been contact with other students of \gls{TDT4250}.
The main reason is that the course is taught in the autumn, while the master thesis is done during the spring semester.
However, the author has been a student of \gls{TDT4250} and talked to other classmates during the semester for their opinion on \acrshort{EMF} and \gls{Eclipse}.

\paragraph{Design Science Research methodologies can cause ad-hoc evaluations}
The literature on Design Science Research says little on how to evaluate.
There are only abstract, high level recommendations to the forms of evaluation, such as action research, case study, simulation and so on.
This can cause a researcher to create an evaluation which is ultimately positive of their view, or disregards important aspects.
The author has attempted to keep the evaluations related to the original research goals, and grounded in existing practice.
Regardless, the weakness of unguided evaluation design is there.

