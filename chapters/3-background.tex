\chapter{Background}\label{chap:background}

This background section will explain some of the concepts, approaches, technologies and software architectures required to understand this thesis.
The findings from the pre-project in \cite{rekstadModelingEnvironmentCloud2020} will also be presented in more detail than the introduction, as the findings are central to this thesis.
Lastly, a section on open source software project management follows, as they shape many of the choices made in the implementation of a solution.

\section{Conceptual Modeling and Model-Driven Development}\label{sec:conceptual-modeling}

\paragraph{Rationale}
\Acrfull{MDD} is the approach to software development which this thesis aims to support.
Therefore, and understanding of \acrshort{MDD} is beneficial, in order to see how an editor should work.

\paragraph{Modeling and abstraction}
The core of \acrshort{MDD} is the model.
The model is a human created construct, formed through humans working together to discuss and refine a problem domain until they reach a consensus of what abstractions help them solve the relevant problems~\cite[p.~154]{brambillaModeldrivenSoftwareEngineering2012}.
Humans perceive the world (and problem domain) as many different phenomena, and conceptual modeling is the act of trying to describe these at some level of abstraction~\cite[p.~1,408]{krogstieModelbasedDevelopmentEvolution2012}.
The model is assumed to resemble the phenomena and work the same way, and yet be simpler than the real world~\cite[p.~414]{krogstieModelbasedDevelopmentEvolution2012}.
Abstraction means to find something common in different observations of a phenomena, and \textit{generalize} their features, \textit{classify} coherent clusters of objects and \textit{aggregate} concepts into more complex ones~\cite[p.~1]{brambillaModeldrivenSoftwareEngineering2012}.
The model will never describe every aspect of the world perfectly, but can \textit{reduce} the world down to relevant aspects, and easily \textit{map} between model elements and real world phenomena~\cite[p.~1-2]{brambillaModeldrivenSoftwareEngineering2012}.


\paragraph{Modeling languages}
In order to describe the model, a \textit{language} is used.
To realize the benefits of \acrshort{MDD}, a \textit{formal language} is used.
The language can be textual or graphical, or both, and imposes a formally defined syntax on the modeler~\cite[p.~13]{brambillaModeldrivenSoftwareEngineering2012}.

\paragraph{Modeling tools}
The advantage of using a formal language is that it can be parsed and understood by software tools, as well as humans.
The tools can validate the model according to the syntax, and to specific rules for the domain.
Tools can also generate code, or execute the model itself.
The model can be transformed into other models, or text or graphics~\cite[p.~8]{brambillaModeldrivenSoftwareEngineering2012}.

\paragraph{\acrlong{MDD}}
The central idea of \acrlong{MDD} is that the model is the source of truth that \textit{drives} the rest of the engineering and development~\cite[p.~9]{brambillaModeldrivenSoftwareEngineering2012}.
There is not a separate model for analysis and for design, but a single one for both~\cite[p.~49]{evansDomaindrivenDesignTackling2004}.
The software code becomes an expression of the model itself, and changes to the code often happen as the result of changes to the model~\cite[p.~49]{evansDomaindrivenDesignTackling2004}.
Because the model and the software are so directly related, the \acrshort{MDD} approach is heavily reliant on tools to automate the tasks of validation and code generation.
The formal language may also sacrifice some of its human readability in order to be understood by tools~\cite[p.~232]{krogstieModelbasedDevelopmentEvolution2012}.
To solve this, one can use other tools that interpret, transform or present models in other ways~\cite[p.~233]{krogstieModelbasedDevelopmentEvolution2012}.
This increases the reliance on tools for \acrshort{MDD} even more, including visual editors.




\section{Model-Driven Development at NTNU in the Course TDT4250}\label{sec:tdt4250}

\paragraph{Rationale}
Because the target audience of the software solution (tree editor) are students at \acrshort{NTNU}, it is helpful to know how they work with \acrlong{MDD}.
Their use cases are the ones being solved, meaning the solution must be made with this context in mind.

\paragraph{\Acrshort{MDD} at \acrshort{NTNU}}
To do \acrlong{MDD} effectively, tools should be used.
In the course ``\textit{\gls{TDT4250} Advanced Software Design}''\footnote{Course description is available at \href{https://www.ntnu.edu/studies/courses/TDT4250\#tab=omEmnet}{https://www.ntnu.edu/studies/courses/TDT4250\#tab=omEmnet}.} at \acrshort{NTNU}, the chosen tools are in the \acrfull{EMF}~\cite{hallvardtraettebergEMFTDT4250NTNU2017}.
This includes the modeling language \gls{Ecore}, visual editors in \gls{Eclipse}, model validation logic, the code generator named ``GenModel''\footnote{The code generator is actually named ``codegen'', but users only see the configuration model called ``GenModel''.} (generator model), and more.
\Acrshort{EMF} is a battle-tested technology also used in certain industries, and is well integrated with the \gls{Eclipse}.
The course \gls{TDT4250} also uses \gls{Eclipse} as a case study for other software design concepts, such as modularity (plugin architecture) and dynamic systems (OSGi), and custom \acrlongpl{DSL} which automatically work with \gls{Eclipse}.
\Acrshort{EMF} is relevant for most or all of those concepts.

\paragraph{Development methodology}
Students are taught a methodology or approach for how to do modeling.
They start by specifying a problem space, for example bookkeeping an organization of employees or the courses in \acrshort{NTNU}, and then abstract the problem into a model.
The initial model is externalized as \gls{Ecore} by using a tree editor in \gls{Eclipse}.


Then an \textit{model instance} is made, based on the model, and filled with example data from the domain.
This model instance is used to test and verify that the model is appropriate for the problem space.
Adjustments are made to the model to accommodate any problems with the model instance.


Then validations can be created for the model, by one or both of the following approaches: writing \acrfull{OCL} into model annotations, or marking the model element with an annotation and implementing it as java code.
\Acrshort{OCL} is a \acrlong{DSL} for navigating models and evaluating expressions, and the \gls{Eclipse} can detect annotations with \acrshort{OCL} and evaluate them against the \gls{Ecore} model.
The other option, writing java code, requires the student to first create a new \textit{genmodel} file from the model (by using a menu in \gls{Eclipse}), generating a java code project from the model, and then writing validation logic into the generated code.
For the java code to be picked up, \gls{Eclipse} can start a new instance which installs the generated code as a plugin~\cite{hallvardtraettebergConstraintsValidationTDT42502020}.


Next up, when the model is deemed sufficient, and the most important validations are in place, the student can try to create a user interface.
One of several choices here is to create an \textit{\gls{Eclipse} plugin}.
\Acrshort{EMF} provides code generation for utilities used to integrate the model into an editor for \gls{Eclipse}.
The student uses the genmodel to create these, and tweaks the code if wanted.
Then everything is installed into \gls{Eclipse} by launching a new \gls{Eclipse} instance with the code installed as a plugin.


Lastly, the user interface can be tested.
The student creates a new model instance file, enters some example data from the domain, and runs validation logic.

\paragraph{Lecture materials}
The steps mentioned in the methodology above are available online in \cite{hallvardtraettebergEMFStepbystepTDT42502017,hallvardtraettebergConstraintsValidationTDT42502020,hallvardtraettebergEditingEcoreModel2017,hallvardtraettebergGenmodelTDT4250NTNU2017}.
This is an advantage, because they can by used used in this master's thesis as a basis for creating evaluations and acceptance criteria.



\section{Eclipse Modeling Framework Editors for Ecore}\label{sec:emf-editors}

\paragraph{Rationale}
These editors are the ones being re-implemented in \gls{cloud}-based \acrshortpl{IDE}.
Understanding their functionality and workings is important, as these editors shape the work of this thesis.
The functionalities provided are assumed highly usable and good, because they are the result of many years of work and experience.
This allows this thesis to skip the work of doing usability testing with regards to feature design, as long as the features are similar enough to the copied ones.

\paragraph{Multiple editors}
When editing \gls{Ecore} models in \gls{Eclipse}, there are different editors to pick from.
Usually, \gls{Ecore} models and model instances are saved as \acrfull{XMI}, which is a standardized serialization format based on XML.
The \gls{Ecore} models have the file extension \texttt{.ecore} while model instances either have \texttt{.xmi} or a custom extension for the model, specified by the modeler (e.g. \texttt{.organization} or \texttt{.courses}).
The GenModel has \texttt{.genmodel} as file extension.
However, \gls{Ecore} models are rarely (if ever) edited as XML.
Instead, the files are loaded and presented in a tree structure editor or diagram editor.
These editors are specialized for \gls{Ecore}, and can understand the model.


The diagram based editors use a notation that is based on \gls{UML} Class Diagrams, with boxes, labels and arrows.
Which editor to use can often be a personal preference.
They are all functionally equivalent, with regards to modeling.
The next subsections will describe the most common tree editors in more detail.

\subsection{Sample Reflective Ecore Model Editor}\label{sec:sample-reflective-editor}

The ``\textit{Sample Reflective Ecore Model Editor}'' is one of the main \gls{Ecore} editors in \gls{Eclipse}.
A screenshot of the editor is shown in \cref{fig:sample-reflective-ecore-model}.
The model instances can be edited in a \textit{reflective} editor (without the user first generating java code and installing an \gls{Eclipse} plugin).
Here, reflective means that the editor uses a metamodel (see \cref{sec:emf-metamodel}) for the model instance, and tries to infer the tree structure from containment relationships.


This editor can open both \gls{Ecore} models and model instances.
A screenshot of a model opened in the editor is shown in \cref{sfig:sample-reflective-ecore-model-screenshot}, and a model instance in \cref{sfig:sample-reflective-ecore-model-instance-screenshot}.


This editor is \gls{open source}%
\footnote{Sample Reflective editor source: \href{https://git.eclipse.org/c/emf/org.eclipse.emf.git/tree/plugins/org.eclipse.emf.ecore.editor}{\nolinkurl{https://git.eclipse.org/c/emf/org.eclipse.emf.git/tree/plugins/org.eclipse.emf.ecore.editor}}.}%
, and the editor is itself originally generated by a genmodel~\cite[p.~10]{rekstadModelingEnvironmentCloud2020}.


This editor internally uses a java class called \texttt{ReflectiveItemProvider}%
\footnote{\texttt{ReflectiveItemProvider} source code: \href{https://git.eclipse.org/c/emf/org.eclipse.emf.git/tree/plugins/org.eclipse.emf.edit/src/org/eclipse/emf/edit/provider/ReflectiveItemProvider.java}{\nolinkurl{https://git.eclipse.org/c/emf/org.eclipse.emf.git/tree/plugins/org.eclipse.emf.edit/src/org/eclipse/emf/edit/provider/ReflectiveItemProvider.java}}}
from the \texttt{org.eclipse.emf.edit} \acrshort{EMF} package, to extract text labels and infer icons for the tree view~\cite[p.~10]{rekstadModelingEnvironmentCloud2020}.


For \gls{Ecore} models (with \texttt{.ecore} file extension, not model instances), it uses an \texttt{EcoreItemProviderAdapterFactory}%
\footnote{\texttt{EcoreItemProviderAdapterFactory} source code: \href{https://git.eclipse.org/c/emf/org.eclipse.emf.git/tree/plugins/org.eclipse.emf.ecore.edit/src/org/eclipse/emf/ecore/provider/EcoreItemProviderAdapterFactory.java}{\nolinkurl{https://git.eclipse.org/c/emf/org.eclipse.emf.git/tree/plugins/org.eclipse.emf.ecore.edit/src/org/eclipse/emf/ecore/provider/EcoreItemProviderAdapterFactory.java}}}
to get labels and icons~\cite{edmerksEcoreEditorJava2021}.


These ``item providers'' are especially interesting, because they could be reused in a new editor.

\begin{figure}
    \centering
    \begin{subfigure}[b]{.45\textwidth}
        \centering
        \includegraphics[width=\textwidth]{figures/pre-project/ecore-sample-reflective-ecore-model-editor}
        \caption{A model opened in the editor.}\label{sfig:sample-reflective-ecore-model-screenshot}
    \end{subfigure}
    \hfill
    \begin{subfigure}[b]{.45\textwidth}
        \centering
        \includegraphics[width=\textwidth]{figures/pre-project/ecore-sample-reflective-ecore-model-editor-instance.png}
        \caption{A \emph{dynamic instance} (\acrshort{XMI} file) opened in the editor.}\label{sfig:sample-reflective-ecore-model-instance-screenshot}
    \end{subfigure}
    \caption{Screenshots of the Sample Reflective Ecore Model Editor in \gls{Eclipse}.}\label{fig:sample-reflective-ecore-model}
\end{figure}


\subsection{EMF Forms Ecore Editor}\label{sec:emfforms-editor}

The \textit{EMF Forms Ecore Editor} is a newer editor than the Sample Reflective editor, and uses EMF Forms\footnote{More info about EMF Forms here: \href{https://www.eclipse.org/ecp/emfforms/index.html}{\nolinkurl{https://www.eclipse.org/ecp/emfforms/index.html}}.} as the technology to provide a user interface~\cite{eclipsesourceEMFFormsEditors2016}.
This editor is \gls{open source}%
\footnote{\textit{EMF Forms} source code: \href{https://git.eclipse.org/c/emfclient/org.eclipse.emf.ecp.core.git/tree/bundles/org.eclipse.emfforms.editor.ecore}{\nolinkurl{https://git.eclipse.org/c/emfclient/org.eclipse.emf.ecp.core.git/tree/bundles/org.eclipse.emfforms.editor.ecore}}.}.
A screenshot of the editor is shown in \cref{fig:emf-forms-ecore-editor}.


This editor is implemented as a generic editor for all \gls{Ecore} model instances, and two subclasses that are specialized for \gls{Ecore} and GenModel~\cite{eclipsesourceEMFFormsEditors2016}.
The generic editor is called \textit{Generic XMI Editor} in \gls{Eclipse}, and the \gls{Ecore} specific editor is called \textit{Ecore Editor}.


The biggest difference compared to the Sample Reflective editor, is how the user interface looks, and that the property sheet is customized based on a \textit{view model file}.
The Sample Reflective editor uses \gls{Eclipse}'s built in property panel.
In the EMF Forms editor, the properties are also grouped into \textit{standard} and \textit{advanced}.

\begin{figure}[htbp]  % order of priority: h here, t top, b bottom, p page
  \centering
  \includegraphics[width=\textwidth]{figures/pre-project/ecore-eclipse-emf-forms-model-editor.png}
  \caption[EMF Forms Ecore Editor]{A screenshot of a model in the EMF Forms based Ecore Editor.}\label{fig:emf-forms-ecore-editor}
\end{figure}

\iffalse{} 
{
%Skip diagrams for now. Tree editors are more important.

  \subsection{Ecore Tools diagrammatical editor}\label{sec:ecore-tools-editor}
  
  The \textit{Ecore Tools} editor presents \gls{Ecore} as class diagrams, similar to \gls{UML} Class Diagrams.
  
  \subsection{EMF.Cloud ecore-glsp diagrammatical editor}\label{sec:ecore-glsp-editor}
  %TODO
  
}
\fi


%* Eclipse EMF offers different Ecore editors. Tree-editor is important for developers, and diagrams are important in runtime and end-users.



\section{Introduction to Tree Structures}\label{sec:tree-structures}

\paragraph{Rationale}
Because the editors center around a tree structure, a clear understanding of trees is helpful.

\paragraph{Trees}
A \textit{tree} is a data structure.
The tree is composed of \textit{nodes}, and one node is designated as the \textit{root node} node or \textit{tree root}.
Each node can have zero or more \textit{children} nodes, and one \textit{parent} node.
The root node does not have a parent.
When representing the tree as code, it is possible to omit either the parent or child relationship in a node, making the parent or child implicit.
The relationship can still be found, by \textit{traversing} the tree.
Traversing means to visit every node it the tree by following the parent or child relationships.
Nodes that are children of the same parent are called \textit{siblings}, and parents of parents are called \textit{grandparents}.

\paragraph{Visualizing trees}
There are many ways to present trees to humans.
Two common approaches are \textit{hierarchy} and \textit{diagram}.\\

In a hierarchy, the parent is presented as a row, and its children on separate rows below (see \cref{sfig:tree-visualized-hierarchy}).
The children are often indented as well, and possibly connected with dots or lines to the parent.\\

In a diagram, nodes are often displayed as a circle or box (see \cref{sfig:tree-visualized-diagram}).
The parent is displayed above its children, and the children are aligned on the same row.
The parent-child relationship is shown as a line or arrow, connecting the parent to the child.

\begin{figure}[htbp]
    \centering
    \begin{subfigure}[b]{.45\textwidth}
        \centering
        \includegraphics[width=\textwidth]{figures/tree-hierarchy.png}
        \caption{A tree visualized as a hierarchy. The top node is the root.}\label{sfig:tree-visualized-hierarchy}
    \end{subfigure}
    \hfill
    \begin{subfigure}[b]{.45\textwidth}
        \centering
        \includegraphics[width=\textwidth]{figures/tree-diagram.pdf}
        \caption{A tree visualized as a diagram. The blue node at the top is the root.}\label{sfig:tree-visualized-diagram}
    \end{subfigure}
    \caption[Tree Structure Visualizations]{A tree visualized as a hierarchy and diagram. The labels are section titles of \cite{rekstadModelingEnvironmentCloud2020}, as an example.}\label{fig:tree-visualized}
\end{figure}

\paragraph{Nodes}
The tree is more useful when the nodes have properties.
The minimum property is children or parent.
But a useful property is a name, label or id, with regards to presenting the tree to a human.
There may be properties on the relationships between a node and its children, but these may be hard to present visually in hierarchy-type visualizations.
For a diagram type visualization, the properties may be presented as labels on the edge.

\paragraph{Mapping to trees}
A data structure can be mapped to a tree if it has separate objects with a references, containment or aggregation relationship.
The references can not be circular (where a node has a child which is also a parent or grandparent etc.).
There can be different ways to map to a tree, depending on what properties are used (or not used).
The labels can also come from various object properties, be derived from them or combine multiple properties into one label.

\paragraph{Editing a tree}
Common operations on trees either modify the structure, or modify the properties of a node.
Structural modifications can be to add a new child, to delete a child, or to move a child from one parent to another.
Nodes can be copied, and pasted on the same parent or other parents, or themselves.
Less common operations are inserting a new node between a parent and child, turning the latter into a grandchild.
Likewise, a node can be removed, merging its children into its parent, making them effectively siblings to the removed node.



\section{Master-Detail Tree Editor}\label{sec:master-detail}

\paragraph{Rationale}
The tree editors use a layout pattern called \textit{master-detail}.

\paragraph{Description}
As the name \textit{Tree Editor} implies, they are used to edit a tree.
There are mainly two different things that can be edited: the parent-child relationships and the node's properties.
The user interfaces for the tree editors in \cref{sec:emf-editors} use a pattern called \textit{master-detail}.
This means the user interface is composed of two parts: a \textit{master view} and a \textit{detail view}.


\paragraph{Master view}
The tree structure is shown as a hierarchy in the master view.
It is common for the master view to be positioned to the left of a detail view, or above it.
The user interacts with the master view to add, remove and select nodes.
Adding a new child to a parent is done here.


\paragraph{Detail view}
When a node is selected, its properties are displayed in the detail view.
It is common for the detail view to be positioned to the right of a master view, or below it.
The detail view is usually a \textit{input form} or tabular (rows and cells) structure.
The user usually enters text, numbers, ticks checkboxes and opens selection dialogues from the detail view.


\section{An Overview of EMF:\ Ecore Metamodel, XMI Serialization and GenModel for Code Generation}\label{sec:emf-metamodel}


\paragraph{Rationale}
The \acrfull{EMF} is the \acrlong{MDD} framework used in \gls{TDT4250}.
The tree editor will modify \gls{Ecore} models, so it helps to understand the concepts and names used in the \gls{Ecore} metamodel.
It is also useful to know the different tools and components in \acrshort{EMF}, because the tree editor intends to reuse as much of them as possible internally, to save development effort.

\paragraph{\acrlong{EMF}}
The \acrfull{EMF} is a part of the Eclipse Modeling project from the Eclipse Foundation.
It is a framework and code generation facility that lets developers define models.
The models can be java code, \gls{XMI} or \gls{UML}, and the other two can be generated~\cite[p.~14]{edmerksEMFEclipseModeling2009}.
This framework may be chosen as the tools for doing \acrlong{MDD} (see \cref{sec:conceptual-modeling}).
In EMF, the models are expressed with the \gls{Ecore} modeling language.
This modeling language is similar to \gls{UML} Class Diagrams, in terms of the concepts and what it can express~\cite[p.~16]{edmerksEMFEclipseModeling2009}.
The real world data that could fit inside a specific model is called a \textit{model instance}.

The framework was made to take use of the editing capabilities and utility of the \gls{Eclipse}~\cite{edmerksEMFEclipseModeling2009}.
This means that there is much tooling and integration for \acrshort{EMF} with \gls{Eclipse}.
For example, EMF can generate a plugin to edit model instances in \gls{Eclipse}.

%TODO
% eclipse editor plugins, language, codegen, ocl, serialization format

\paragraph{\Gls{Ecore} metamodel}
The modeling language in \acrshort{EMF} is \gls{Ecore}.
A \textit{metamodel} is the model of a model.
This means that Ecore is the metamodel for all models expressed using \acrshort{Ecore}.
Ecore is itself modeled in Ecore, so it is its own metamodel.


\paragraph{Model concepts}
The main concepts used in \gls{Ecore} to model, are \texttt{EClass}, \texttt{EAttribute}, \texttt{EReference} and \texttt{EDataType}%
\footnote{The name Ecore comes from EMF Core, and the `E' prefix for \texttt{EClass} etc.\ come from Ecore.}.
These are distinct objects with names, properties and inheritance, like in object oriented programming.
As for the metamodel, \texttt{EClass}, \texttt{EAttribute} and \texttt{EReference} are all extending \texttt{ENamedElement}, which defines their \texttt{name} property~\cite{edmerksEMFEclipseModeling2009}.

When modeling, \texttt{EClass} is used to create java classes.
The \texttt{EAttribute} and \texttt{EReference} are used to model class properties, like member variables.
An \texttt{EAttribute} defines a property, such as e.g. \textit{age} or \textit{address}, while \texttt{EReference} defines a reference/association to another \texttt{EClass}, e.g. \textit{parent} or \textit{order}.
The \texttt{EAttribute} has a attribute type, the \texttt{EDataType}, which can be e.g. \texttt{EInt} or \texttt{EString}~\cite{edmerksEMFEclipseModeling2009}.

Java class methods are modeled with another concept, the \texttt{EOperation}.
Lastly, everything in the model lives inside an \texttt{EPackage}, which represents a java package (or other kind of code module).
There are more concepts in \gls{Ecore}, but many are only used internally as part of the metamodel, to represent \gls{Ecore} itself.


\paragraph{\Acrshort{XMI} serialization}
When an \gls{Ecore} model is written as a text file, it needs \textit{serialization}.
The official format for serializing Ecore is \acrfull{XMI}.
This format is based on \acrfull{XML}.
Model instances can also be serialized as \acrshort{XMI}.
It is also possible to serialize \gls{Ecore} to other formats, like \gls{JSON}, using third party tools.
% serialization, standardized EMOF, default for ecore, XML,

\paragraph{\acrshort{EMF} java \acrshort{API}}
The java code generated by \acrshort{EMF} will by default extend a set of java classes defined by \acrshort{EMF}.
Instead of a generated \texttt{EClass} extending \texttt{java.lang.Object}, it extends \texttt{EObject}.
And instead of using an \texttt{ArrayList}, a collection in \gls{Ecore} will use a \texttt{EList}.
When creating a new instance, the class constructor is not used, but a Factory instance on the generated \texttt{EPackage} for the model.


All of these framework java-classes are the \acrshort{EMF} java \gls{API}.
They provide much of the power, flexibility, reflection and meta-modeling capabilities of \acrshort{EMF} in java.
For example, a program can work with a \acrshort{EMF} model without knowing the code beforehand, by using the reflection \acrshort{API} to retrieve names and properties of a model object.


The \acrshort{API} also provides utilities for working with the model.
There are \acrshortpl{API} for listing the children of an \texttt{EObject}, getting a human representation of it, and for modifying and observing state changes.
Another important \acrshort{API} is the \texttt{ResourceSet} and \texttt{Resource}, used to read and save models to serialized \acrshort{XMI} files.

\paragraph{GenModel code generation}
Code generation is an important part of \acrshort{EMF}.
The generator can be configured with its own generator model, nicknamed the \textit{GenModel}.
This model holds options for how the code will be named, what templates should write the code, if the code can use the \acrshort{EMF} \acrshortpl{API}, and more.


The generator can also produce more than just a java representation of the model.
A test suite can be generated, with an empty test skeleton for the generated code.
It can also generate utilities for creating model editors, in what is called the \textit{.edit} java package.
The name ``.edit'' is appended to the original package name.
This has \textit{ItemProvider} classes which helps an editor to find the human representations, properties, child objects, and to notify on changes.


Another utility is related to the \gls{Eclipse}, which is the \textit{.editor} java package.
This holds key classes for integrating with \gls{Eclipse}, making it a custom editor.
For example, custom actions, project wizards, eclipse plugin logic is part of this.


\paragraph{Custom code}
The generated code must usually be modified by a developer.
This can be to fill in the implementation of a \texttt{EOperation}, or tweak some behavior.
The generated code has a \texttt{@Generated} java annotation, which the developer changes to prevent the code generator from overwriting the method body.


\section{Visual Studio Code and Theia}

The two \acrshortpl{IDE} relevant for this thesis are Visual Studio Code (\gls{VSCode}) and \gls{Theia}.
Both are available as editors in \gls{Gitpod} as cloud based \acrshortpl{IDE}.

\subsection{Visual Studio Code}

\Gls{VSCode} is a very popular \gls{open source} \acrshort{IDE} created by Microsoft~\cite{stackoverflowStackOverflowDeveloper2019}.
A screenshot is shown in \cref{fig:vscode-ui}.
It uses web technologies like javascript, \gls{Nodejs} and \gls{Electron} to provide an advanced text editor and tools for programming on a desktop.
Originally made only for desktop, \gls{VSCode} was later adapted to also work in a browser when \gls{GitHub}\footnote{GitHub is owned by Microsoft.} launched Codespaces~\cite{svenefftingeProductRoadmapQ1}.
\Gls{VSCode} is extensible, and allows third party developers to create extensions.
These are distributed from Microsoft's extension store: Visual Studio Marketplace\footnote{Marketplace website: \href{https://marketplace.visualstudio.com/vscode}{\nolinkurl{https://marketplace.visualstudio.com/vscode}}.}.

\paragraph{Programming languages}
One common use of extensions is to support new programming languages.
The text editor in \gls{VSCode} is a generic text editor component called \textit{Monaco}~\cite{benjaminpaseroSourceCodeOrganization2020}.
This same text editor is used for all programming languages.
For the text editor to know the keywords, suggestions and other specifics of a programming language, the extension uses a standardized protocol to inform Monaco.
This protocol is called the \acrlong{LSP}, and is described in \cref{sec:lsp}.

\begin{figure}[htbp]  % order of priority: h here, t top, b bottom, p page
  \centering
  \includegraphics[width=\textwidth]{figures/pre-project/vscode-ui.png}
  \caption[VSCode User Interface]{The \gls{VSCode} user interface, annotation with the different components (A-E).}\label{fig:vscode-ui}
\end{figure}

\subsection{Theia}

Theia is based on the \gls{open source} components from \gls{VSCode}, without a proprietary component that Microsoft added for telemetry.
Theia is managed by the Eclipse Foundation under the \textit{Cloud Development project} (see \cref{sec:emf-in-cloud}), and was created to be web based from the start (before Codespaces launched, when \gls{VSCode} was desktop only).
A screenshot is shown in \cref{fig:theia-ui}.
The main uses of Theia are workspace services like \gls{Gitpod} and Eclipse Che, but it is also intended to be a ``web based version'' of the Eclipse Rich Client Platform.
This means tools can create their own distribution of Theia, where they are deeply integrated~\cite{helmingEclipseTheiaIDE2019a}.

\begin{figure}[htbp]  % order of priority: h here, t top, b bottom, p page
  \centering
  \includegraphics[width=\textwidth]{figures/pre-project/theia-screenshot.png}
  \caption[Theia User Interface]{The \gls{Theia} user interface.}\label{fig:theia-ui}
\end{figure}

\paragraph{Extensions}
Theia can load extensions using the same \acrfull{API} as \gls{VSCode}.
Theia calls these ``Theia Plugins''.
Another way to extend Theia is using ``Theia Extensions''.
These have full control over the \acrshortpl{IDE}, and can modify practically anything.
Installing a Theia Extension requires the user to perform a full compilation of \gls{Theia} itself~\cite{helmingHowAddExtensions2019}.
A Theia Plugin (or \gls{VSCode} extension) however, can be installed at runtime.
Because of licensing issues with Microsoft and the Visual Studio Marketplace, Theia Plugins are instead hosted at a independent marketplace called \textit{OpenVSX}~\cite{svenefftingeOpenVSX2020}.


\section{Visual Studio Code's Custom Editor API}\label{sec:vscode-custom-editor}

A \gls{VSCode} extension is allowed to use a set of \acrlongpl{API} provided by \gls{VSCode}.
One such \acrshort{API} is the \textit{Custom Editor API}.
This allows an extension developer to create \textbf{custom editors other than text editors}.
This could be diagrams, pictures, graphs, or \textbf{trees}, for example.
The developer has the full freedom of a web browser, as they are given their own isolated frame.
Normally, an extension cannot modify the user interface outside of the provided \acrshortpl{API}.
This is in contrast to inside the provided \texttt{WebView}, where the developer has to \textit{create and manage} the entire user interface.
In addition to a user facing \texttt{WebView}, the developer must create their own document model.
By default, \gls{VSCode} uses a document model for text documents, with selections, edits, versions and more.
The \texttt{CustomDocument} only has a uri pointing to the file.
Another central part is the \texttt{CustomEditorProvider}, with a few methods to fill in, like opening, undoing and saving a document.




\section{Language Server Protocol Architecture}\label{sec:lsp}


\paragraph{Goal}
There are many programming languages, and many \acrlongpl{IDE}.
Traditionally, every \acrshort{IDE} would have a special integration for every language it supported.
Extracting tokens, keywords, providing auto completion, code formatting and so on.
This leads to a lot of rework every time a new \acrshort{IDE} comes around, and duplication of work every time a new programming language is supported.
Essentially, every $m$ number of \acrshortpl{IDE} that support an $n$ number of programming languages result in $m\times{}n$ different integrations.
This is illustrated in the left side of \cref{fig:lsp-m-times-n}.\\

A solution to this $m\times{}n$ problem is the \textit{\acrlong{LSP}} (\acrshort{LSP}).
If instead, every \acrshort{IDE} has a generic text editor for all languages, they only need to support the \acrshort{LSP}.
Once an editor ``talks'' \acrshort{LSP}, it can support \textbf{all programming languages} that have a \acrshort{LSP} \textit{language server}.
Likewise, a programming language only needs to develop \textbf{one language server}, and it supports all \acrshortpl{IDE} that use \acrshort{LSP}~\cite{microsoftOverview}.
This is shown in the right side of \cref{fig:lsp-m-times-n}.\\

This protocol was created by Microsoft, and is in use today on \gls{VSCode}.
Many \acrshortpl{IDE} and text editors have adopted it afterwards, like \gls{Eclipse} (with LSP4E), Atom, Vim, Sublime Text, Spyder and more, via both official and unofficial plugins to these \acrshortpl{IDE}~\cite{microsoftToolsSupportingLSP}.
The protocol is quite extensive, and defines approximately 40 different requests with corresponding responses, 20 notification types, in addition to data structures needed to support all of these~\cite{microsoftLanguageServerProtocol2021}.

\begin{figure}[htbp]  % order of priority: h here, t top, b bottom, p page
  \centering
  \includegraphics[width=\textwidth]{figures/pre-project/lsp-languages-editors.png}
  \caption[The Language Server Protocol Benefits]{The benefits of using the \acrshort{LSP}. The left side shows all the integrations (as arrows) required for 3 languages (javascript, python, java) and 3 editors (VSCode, Atom, Vim), without the \acrshort{LSP}.
  The right side shows how the \acrshort{LSP} can reduce the amount of work by unifying the common elements of programming language editors into a standard protocol. Figure copied from~\textcite{microsoftLanguageServerExtension2020}.}\label{fig:lsp-m-times-n}
\end{figure}

\paragraph{Protocol}
The \acrlong{LSP} is based on a \textit{Base Protocol}.
This Base Protocol is similar to HTTP, in that it has a \textit{header} section and a \textit{content} section.
The content section contains \acrfullpl{RPC}, using a protocol called \gls{JSON-RPC}.
This is shown in \cref{fig:lsp-architecture}.

\begin{figure}[htbp]  % order of priority: h here, t top, b bottom, p page
  \centering
  \includegraphics[width=\textwidth]{figures/pre-project/lsp-protocol.png}
  \caption[LSP Protocol Design]{The \acrlong{LSP} protocol extends a Base Protocol with JSON-RPC content.}\label{fig:lsp-architecture}
\end{figure}

\subsection{Base Protocol}\label{sec:base-protocol}

All communication in \acrshort{LSP} uses concepts from the Base Protocol.
This protocol has a header and content section, as mentioned above.
Conceptually, the protocol assumes there is one \textit{client} and one \textit{server} which communicates.
Note that the server can also initiate requests to the client.
In addition, the Base Protocol defines specific types of messages: \textit{Request Message}, \textit{Response Message}, \textit{Notification Message}, and \textit{\$ Notifications and Requests}~\cite{microsoftLanguageServerProtocol2021}.

\paragraph{Header}
The header is comparable to a HTTP header, with key-value pairs separated by colon, and a line break for each new pair.
The currently supported header keys are \texttt{Content-Length} and \texttt{Content-Type}.
The \texttt{Content-Length} specifies how many bytes the content is~\cite{microsoftLanguageServerProtocol2021}.

\paragraph{Content}
The content section contains the actual message data, like requests and responses.
This section follows the \gls{JSON-RPC} protocol, described later in \cref{sec:json-rpc}~\cite{microsoftLanguageServerProtocol2021}.

\paragraph{Request and Response}
A Request Message describes a request from a client to the server.
This must have an ID, a method name (for \gls{RPC}) and parameter values for the method.
When a client sends a Request, it means that the server should execute the given method with the given parameters.
The server must then respond with the results of the execution in a Response Message.
This Response must have the id of the originating Request, as well as the results or an error~\cite{microsoftLanguageServerProtocol2021}.\\

An example of a Request is shown in \cref{lst:lsp-example}.
It is the \texttt{textDocument/signatureHelp} method, specifying a \texttt{textDocument} and \texttt{position} with parameter values for the \texttt{textDocument/signatureHelp} method call.

\begin{lstlisting}[caption={A Request Message Example}, label={lst:lsp-example}]
Content-Length: 201

{
    "jsonrpc":"2.0",
    "id":"1",
    "method":"textDocument/signatureHelp",
    "params": {
        "textDocument": { "uri": "file:/..." },
        "position": { "line": 5, "character": 3 },
    }
}
\end{lstlisting}


\paragraph{Notification}
A Notification Message is more like an event.
It does not have an ID, and does not get a Response Message in return.
The Notification, like the Request, specifies a method and parameter values~\cite{microsoftLanguageServerProtocol2021}.

\paragraph{\$ Notifications and Requests}
If a Notification or Request has a \lstinline{$/} at the start of the method name, it is an optional and protocol implementation-specific message.
Not all clients and servers handle these messages.
A notification can be ignored, and a request must be answered with a specific error, if the message is not implemented.

\subsection{Language Server Protocol}
The \acrfull{LSP} defines \gls{JSON-RPC} requests, response and notification messages that are sent in the Base Protocol.
These are specified as method names and parameter values, as well as semantics and rules related to the sequences, responses to, and content of these messages.
\acrshort{LSP} also defines a set of \gls{JSON} data structures, which are used in the messages as parameter values and response types~\cite{microsoftLanguageServerProtocol2021}.
The protocol is versioned, where \texttt{3.16} is the current version.\\

The \acrshort{LSP} defines many messages, related to these categories: 
\begin{itemize}
  \item Window
  \item Telemetry
  \item Client
  \item Workspace
  \item Text Synchronization
  \item Diagnostics
  \item Language Features
\end{itemize}
The most important category is Language Features, which define Requests such as: completion, hover, signature help, references, code action, formatting, rename, and more.
The full list is available in the \citetitle{microsoftLanguageServerProtocol2021}~\cite{microsoftLanguageServerProtocol2021}.



\section{JSON-RPC}\label{sec:json-rpc}


\Gls{JSON-RPC} is a stateless and lightweight protocol for doing \acrlongpl{RPC} (\acrshort{RPC}).
It works over any transport mechanism that can send and receive text.
The data in \gls{JSON-RPC} is sent as \gls{JSON}, an object structure serialization format originally from javascript~\cite{json-rpcworkinggroupJSONRPCSpecification2010}.\\

\acrshort{RPC} is a technique to start a procedure on a remote server, as the name suggests.
A procedure is synonymous with a function or method in programming.
In \gls{JSON-RPC}, they are called by specifying the name and the parameters in a \textit{Request object}.
A request object must have the properties \texttt{jsonrpc} and \texttt{method}, and \texttt{id} if it is not a notification.
It may have \texttt{params}.
Notifications, as explained for \acrshort{LSP}, do not need a response.
The \textit{Response object} must have the properties \texttt{jsonrpc}, \texttt{id}, and either \texttt{result} or \texttt{error}.
The result is the return value for the called procedure.
The error is an \textit{Error object}, with \texttt{code}, \texttt{message} and \texttt{data}~\cite{json-rpcworkinggroupJSONRPCSpecification2010}.\\

The request parameters in \texttt{params} are either a list of positional parameters, or an object with named parameters.
If there are no parameters, they can be omitted~\cite{json-rpcworkinggroupJSONRPCSpecification2010}.
An example of several \gls{JSON-RPC} messages are shown in \cref{lst:json-rpc-example}.
The arrows indicate the direction.

\begin{lstlisting}[caption={JSON-RPC examples copied from \cite{json-rpcworkinggroupJSONRPCSpecification2010}.}, label={lst:json-rpc-example}]
Syntax:
--> data sent to Server
<-- data sent to Client

RPC call with positional parameters:
--> {"jsonrpc": "2.0", "method": "subtract", "params": [42, 23], "id": 1}
<-- {"jsonrpc": "2.0", "result": 19, "id": 1}

RPC call with named parameters:
--> {"jsonrpc": "2.0", "method": "subtract", 
     "params": {"subtrahend": 23, "minuend": 42}, "id": 3}
<-- {"jsonrpc": "2.0", "result": 19, "id": 3}

Notifications:
--> {"jsonrpc": "2.0", "method": "update", "params": [1,2,3,4,5]}
--> {"jsonrpc": "2.0", "method": "foobar"}
\end{lstlisting}



\section{\Gls{cloud} and \gls{Gitpod}}

\input{chapters/3-background/11-cloud-gitpod.tex}


\section{\acrlong{EMF} in the \Gls{cloud}}\label{sec:emf-in-cloud}


\paragraph{The Eclipse Cloud Development project}
As mentioned in the introduction (\cref{sec:intro-eclipse-cloud}), the Eclipse ecosystem is interested in running software in the \gls{cloud}.
This means that they have spent the last few years creating tools to support \gls{cloud} oriented deployments for software built on \acrlong{EMF}.
The Eclipse Foundation has an umbrella project called \textit{Eclipse Cloud Development}.
In the Eclipse Foundation, a project is not a single codebase, but rather a home for frameworks, tools and components.
Under this umbrella exists projects like \textit{EMF.Cloud}, \textit{Eclipse Che}, \textit{Eclipse GLSP}, \textit{Eclipse Theia}, \textit{Eclipse OpenVSX} and more~\cite{beatonEclipseCloudDevelopment2014}.

\subsection{EMF.Cloud}
The Eclipse ecosystem found it suitable to create a new project under this umbrella, and called it \textit{Eclipse EMF.Cloud}.
The description for Eclipse EMF.Cloud starts with the following:
\begin{quote}
  ``\textit{Eclipse EMF.cloud comprises a set of components that facilitate and simplify the adoption of the Eclipse Modeling Framework (EMF) in cloud-based applications.\\
  \textelp{}\\
  As a consequence, by its nature, EMF.cloud is open to any software project that aims to address the challenges and specific requirements of using any aspect of EMF in a browser-based setting or cloud deployment.}''\\
  ---~\textcite{smithEclipseEMFCloud2019}
\end{quote}

\paragraph{EMF.Cloud software}
The components provided by EMF.Cloud are still in active development.
Most of them center around building a modeling environment in \gls{Theia}, for existing \acrshort{EMF} models.
The example case that is used is a ``Coffee brewing model''.
Because much of the work targets Theia, the Eclipse ecosystem uses Theia Extensions.
This means they can not be used in \gls{Gitpod}, because the \acrshort{IDE} has to be replaced with their customized Theia.
However, much of the work here is still relevant, as components to use in a \gls{VSCode} extension, and as design to draw inspiration from.\\

The EMF.Cloud project currently provides these components, according to \cite{tobiasortmayrEclipseemfcloudEmfcloud2021}:

\begin{itemize}
  \item modelserver
  \item modelserver-theia
  \item model-validation
  \item coffee.editor
  \item ecore-glsp
  \item theia-tree-editor
  \item json-forms-property-view
  \item modelserver-glsp-integration
  \item emf-jackson
\end{itemize}

The most relevant components for this thesis are detailed in the following subsections.

\subsubsection{Model Server}
The EMF.Cloud Model Server provides a web server for working with \acrshort{EMF} models.
While \acrshort{EMF} already support model loading, manipulation and serialization in the \acrshort{EMF} runtime \acrshort{API}, this server exposes these to the web.
It does so by providing a \gls{REST} \acrshort{API} for working with models, and \gls{WebSocket} channels for subscribing to change events.
The Model Server also manages a ``shared editing domain'' for the loaded models, and changes models using \acrshort{EMF} \texttt{Command}s~\cite{foundationEMFCloud}.\\

This Model Server is already used in other EMF.Cloud components, like the coffee editor and ecore-glsp~\cite{eugenneufeldEclipseemfcloudCoffeeeditor2021,ninadoschekEclipseemfcloudEcoreglsp2021}.

\subsubsection{Theia Tree Editor}\label{par:theia-tree-editor}
Theia Tree Editor is a framework for creating master-detail tree editors~\cite{foundationEMFCloud}.
It uses the Theia extension mechanism, and uses core components of \gls{Theia} itself~\cite{rekstadModelingEnvironmentCloud2020}.
This hinders reuse in other \acrshortpl{IDE} like \gls{VSCode}.
However, the data structures and configuration schemas used in the Theia Tree Editor are good sources for design inspiration.
A diagram of its \texttt{Node} interface (for tree nodes) is shown in \cref{fig:theia-tree-editor-nodes}\footnote{Node source code: \href{https://github.com/eclipse-emfcloud/theia-tree-editor/blob/3da9d6a3c58cad140c228408b92a554fe5dd1b41/theia-tree-editor/src/browser/interfaces.ts\#L30}{\nolinkurl{https://github.com/eclipse-emfcloud/theia-tree-editor/blob/3da9d6a3c58cad140c228408b92a554fe5dd1b41/theia-tree-editor/src/browser/interfaces.ts\#L30}}.}
\footnote{SelectableTreeNode source: \href{https://github.com/eclipse-theia/theia/blob/af9b883dd929c79c1593bf4bd526df11600e21cf/packages/core/src/browser/tree/tree-selection.ts\#L109}{\nolinkurl{https://github.com/eclipse-theia/theia/blob/af9b883dd929c79c1593bf4bd526df11600e21cf/packages/core/src/browser/tree/tree-selection.ts\#L109}}.}.


\begin{figure}[htbp]  % order of priority: h here, t top, b bottom, p page
  \centering
  \includegraphics[width=\textwidth]{figures/pre-project/theia-tree-editor-node.png}
  \caption[Class Hierarchy of Theia Tree Editor Nodes]{A class hierarchy of the tree nodes in Theia Tree Editor. Class properties and methods are not shown. The \texttt{Node} interface extends several different interfaces, picking up various properties from each. Only the \texttt{Node} interface is in the Theia Tree Editor library itself. The other interfaces are in the core \gls{Theia} codebase. Adopted from ``Figure 2.7'' in \cite[p.~15]{rekstadModelingEnvironmentCloud2020}}\label{fig:theia-tree-editor-nodes}
\end{figure}

\subsubsection{Coffee Editor}
The Coffee Editor is an example application, trying to demonstrate the use of EMF.Cloud components in a real \gls{cloud} deployment.
This editor uses \gls{Theia}, JSON-Forms, \acrshort{GLSP}, a code generator, and the Model Server~\cite{foundationEMFCloud}.
This editor is interesting because it applies the technologies, demonstrating their use, purpose and value.
It also demonstrates the use of a Model Server shared among multiple editing components, like the GLSP and Theia Tree Editor working on the same backing coffee \acrshort{EMF} model instances.

\subsection{Graphical Language Server Platform (GLSP)}\label{sec:glsp}

This is another project under the Eclipse Cloud Development project.
The \acrfull{GLSP} is a framework for building diagram editors in the web.
The editors can either be standalone or integrated into \gls{Theia} and \gls{VSCode}.
The \acrshort{GLSP} defines its own \acrfull{LSP} for diagrams~\cite{eclipsefoundationGLSP2020}.
A figure from the official website is shown in \cref{fig:glsp-overview}.\\

This is a good source of design inspiration, because it both works with \acrshort{EMF} models, and it applies the \acrlong{LSP} architecture to a new domain other than text editing.

\begin{figure}[htbp]  % order of priority: h here, t top, b bottom, p page
  \centering
  \includegraphics[width=\textwidth]{figures/pre-project/glsp-overview.png}
  \caption[GLSP Overview]{A diagram of GLSP. A web based diagram editor shows the diagram obtained from a Graphical Model Editor Client. This client talks to a graphical model server over the Graphical Language Server Protocol. Copied from \cite{eclipsefoundationGLSP2020}.}\label{fig:glsp-overview}
\end{figure}

\paragraph{GLSP Protocol}\label{par:glsp-actionmessage}
The \acrshort{LSP}-based protocol has a very small ``surface area''.
The protocol uses the same Base Protocol as defined in \acrshort{LSP} (see \cref{sec:lsp}).
The java implementation also uses the same libraries as Eclipse's Java \acrshort{LSP} server: the LSP4J jsonrpc library.
The server side of the GLSP protocol is shown in \cref{lst:glsp-server}.
Compare this to \acrshort{LSP}, which has about 40 Requests and 20 Notifications --- this GLSP protocol only has two requests and one Notification.
Instead of defining many different methods, the ``meat'' of the protocol is inside the \texttt{process} method.
The \texttt{ActionMessage} holds an \texttt{Action}, which is an abstract class with a \texttt{kind} field.
This \texttt{Action} is extended (subclassed) to about 20 different versions.
These have names like \texttt{FitToScreenAction}, \texttt{CenterAction}, \texttt{SelectAction}, \texttt{SaveModelAction}~\cite{tobiasortmayrEclipseglspGlspserverActions2021}.
The GLSP client and server implementations rely on \textit{action handlers}.
The Grphical Modeling Editor Client (usually inside \gls{VSCode} or \gls{Theia}) can decide if an \texttt{ActionMessage} from the diagram viewer should be forwarded to the Graphical Model Server or not.
Sometimes the action can be performed entirely inside the client.
The same applies for messages from the server, which can be forwarded to the diagram editor or stop in the Graphical Model Editor Client~\cite{tobiasortmayrEclipseglspGlspvscodeintegration2021}.

\begin{lstlisting}[language=java, label={lst:glsp-server}, caption={[GLSP Server Interface]GLSP Server java interface. Copied from \cite{philiplangerEclipseglspGlspserver2021}.}]
package org.eclipse.glsp.server.jsonrpc;

import java.util.concurrent.CompletableFuture;

import org.eclipse.glsp.server.actions.ActionMessage;
import org.eclipse.glsp.server.protocol.GLSPServer;
import org.eclipse.glsp.server.protocol.InitializeParameters;
import org.eclipse.lsp4j.jsonrpc.services.JsonNotification;
import org.eclipse.lsp4j.jsonrpc.services.JsonRequest;

public interface GLSPJsonrpcServer extends GLSPServer<GLSPJsonrpcClient> {
   @Override
   @JsonRequest
   CompletableFuture<Boolean> initialize(InitializeParameters params);

   @Override
   @JsonNotification
   void process(ActionMessage message);

   @Override
   @JsonRequest
   CompletableFuture<Boolean> shutdown();
}
\end{lstlisting}

\subsection{Other Tools by the Eclipse Ecosystem}
Not all the efforts by the Eclipse ecosystem are made under the Eclipse Foundation's management.
Some projects exist outside this, in code repositories belonging to individuals and organizations that work with \acrshort{EMF}.
Some of the most relevant software projects are described below.

\subsubsection{JSON-Forms}\label{sec:json-forms}

The purpose of JSON-Forms is to easily create user interfaces for data entry in the web, using HTML forms.
A screenshot of such a form is shown in \cref{fig:json-forms-example}.
JSON-Forms is a project by the EclipseSource organization.
This is the same organization that created the EMF Forms based \gls{Ecore} editor for \gls{Eclipse}, described in \cref{sec:emfforms-editor}.\\

JSON-Forms is using the same core approach as EMF Forms, where the view is described in a declarative fashion with a \textit{UI schema}.
This schema describes a data entry form.
It describes the input fields, their labels, what data they effect and the grouping of view elements~\cite{eclipsesourceWhatJSONForms}.\\

In addition to a UI schema, a form using JSON-Forms needs a \textit{JSON schema}, which describes the types, structure and validation rules for the underlying data.
Together, these two schemas are enough for JSON-Forms to render and edit a \gls{JSON} data object in a user interface.

\begin{figure}[htbp]  % order of priority: h here, t top, b bottom, p page
  \centering
  \includegraphics[width=\textwidth]{figures/pre-project/json-forms-example.png}
  \caption[JSON-Forms Example]{Example of a form rendered with JSON-Forms. Adopted from \cite{eclipsesourceWhatJSONForms}.}\label{fig:json-forms-example}
\end{figure}

\subsubsection{CrossEcore}
CrossEcore is a project by Simon Schwichtenberg with the aim of cross platform code generation using \acrshort{EMF}.
It targets the programming languages C\#, TypeScript, JavaScript, and Swift.
CrossEcore also implements the \acrshort{EMF} runtime \acrshort{API} for these languages, as well as an \gls{OCL} compiler~\cite{simonschwichtenbergCrossecoreEcoretypescript2021}.\\

This project is relevant because it can generate TypeScript code from \gls{Ecore} models, and has also experimented with creating online editors~\cite{simonschwichtenbergCrossEcore}.

\iffalse
% Not relevant enough
\subsubsection{ecore.js}
% TODO
* Under emfjson
\fi



\section{Pre-project Results}

\input{chapters/3-background/13-pre-project-findings.tex}


\section{Open Source Software Project Management and Project Viability}
%TODO

* What can make this Open Source software project viable and worth pursuing further? Anecdotal and empirical evidence, not research.
  * This project needs to live on after the delivery of the Thesis.
  * Correct open source licenses, and "license hygiene" wrt. copy-pasting. Eclipse Foundation do thorough license reviews.
  * Use of programming languages accepted by the developer community.
  * Use of automated build systems accepted by the developer community.
  * Testable code to reduce legacy and maintenance burden.
  * Human readable, clean code. Correct use of Design Patterns. Clean separation of concerns.
  * Use of commonly used and recognized dependencies/libraries/frameworks/tools.
  * CI/CD.
  * Good developer documentation. Architecture diagrams. Informative Readme-files with pictures of the running software. Instructions for developer environment setup.
  * Google Design Documents (?). Not too common in this ecosystem, but valuable inside Google. Complements the readme.
  * Publicly available bug/issue tracker and roadmap.
  * Release management. Semantic versioning. Changelogs. More useful for end-users or those using this as a library/dependency.
  * Specific to Eclipse Foundation, is the "Eclipse Foundation Project Handbook" (https://www.eclipse.org/projects/handbook/) and its checklist.
  * Measures to reduce new-developer onboarding and friction.
