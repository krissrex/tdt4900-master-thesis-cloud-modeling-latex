\chapter{Method}\label{chap:method}


%How to answer the main research question
%TODO

% Design Science, build and evaluate.
\section{Design Science}

* Design Science. Build and evaluate value. Contribute to knowledge base. Is this software we need?

\section{Requirements Engineering}

\subsection{Requirements Extraction}

* Requirements engineering by extraction. Skip user testing for requirement validation. 

\subsection{Source Code Analysis of Similar Projects}
* Source code inspection of similar projects. Use the same architecture, design patterns, build systems, software dependencies. Ensures familiarity for the ecosystem, and uses already empirically validated designs.

\section{Development Methodologies}

\subsection{Agile}

* Agile planning. Avoid big upfront design and specification. Change plans as new discoveries become evident.

\subsection{Iterative Development}
* Iterative development. Work on one component at the time, up to a minimum level of functionality. Then come back later and add more.


\subsection{Lean and Minimum Viable Product}

* Lean development. Create a minimum viable product and see if it works.

\subsection{Domain Driven Design}

* Domain driven design. Increase software quality, developer understanding and software re-use with layered architecture, domain layer and ubiquitous domain language.

\subsection{Test Driven Development}

* Test driven development, where applicable. Verify behavior of critical logic, to reduce bugs and increase developer confidence and speed.

\subsection{Tracer Bullets}
* Tracer bullets. Reduce risk from integration by connecting all the major components, before developing any component fully.

\subsection{Prototyping}

* Pre-project with prototyping. Reduce risk by testing feasibility early.

