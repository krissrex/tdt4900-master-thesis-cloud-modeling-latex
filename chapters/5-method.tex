\chapter{Method}\label{chap:method}

The method used will try to achieve the project objectives with correct results, and avoid or lower risks for project failure.

\paragraph{Pre-project}
The pre-project that came before this master's thesis, in \cite{rekstadModelingEnvironmentCloud2020}, is regarded as a part of the methodology.
It did the initial steps of problem identification, building, and prototyping a solution.

\paragraph{Software project}
Alongside this thesis, \textbf{a software project} will be created, which is developed by the author as part of the method.
A substantial amount of time is dedicated to this project.

\paragraph{General failure criteria}
The project is a failure if the results are invalid, or cannot be realized into a real solution, or are so low quality that the project does not receive further development.
The project is also a failure if it does not provide any value for its stakeholders.

\paragraph{Method overview}
The following sections describe the key elements to the method.
There is an overarching approach, called Design Science Research.
It has 6 phases, from problem identification, to development, to evaluation and communication.
There is no methodology given by Design Science Research for executing the development phase.
Therefore, a method for this phase must be crafted from experience and existing practice.
The development phase consists of requirements engineering methods, and software development methods.


\section{Design Science Research}
%* Design Science. Build and evaluate value. Contribute to knowledge base. Is this software we need?


Design Science Research in information systems is a methodology for creating new knowledge by designing, building and evaluating software \glspl{artifact}.
It may not be as widely known as ``the scientific method'' is, and is therefore explained in more detail.

\paragraph{Design}
\textit{Design} in information systems is an iterative process and a resulting software artifact. A software artifact is to be built to solve problems for humans, and evaluated to prove it solves the problems~\cite[p.~2]{alanhevnerDesignResearchInformation2010}.


\paragraph{Research}
\textit{Research} is an activity that adds new knowledge and understanding about something.
Research should be systematical and use data to answer questions, solve problems and provide understanding~\cite[p.~2,3]{alanhevnerDesignResearchInformation2010}.

\paragraph{Design Science Research}
\textit{Design Science Research} is an approach to research where knowledge is created by design.
It is defined by \textcite[p.~5]{alanhevnerDesignResearchInformation2010} as follows:

\begin{quote}
  \textit{``Design science research is a research paradigm in which a designer answers questions relevant to human problems via the creation of innovative artifacts, thereby contributing new knowledge to the body of scientific evidence.
  The designed artifacts are both useful and fundamental in understanding that problem.''}
\end{quote}

The end goal of a Design Science Research project is to create information technology \glspl{artifact}, that improve exiting solutions or solve a problem for the first time~\cite[p.~6]{alanhevnerDesignResearchInformation2010}.
A similar methodology may also be known under the name \textit{Design and Creation}, as presented by \textcite[p.~108]{oatesResearchingInformationSystems2006}.
According to \textcite{alanhevnerDesignResearchInformation2010}, the artifacts are generally classified as \textit{constructs}, \textit{models}, \textit{methods}, \textit{instantiations} or \textit{better design theories}%
\label{par:artifact-classes}%
\footnote{This thesis aims to produce an instantiation: an implemented or prototype system. The thesis also seeks to advance on \textit{better design theories}, with regards to software architecture and protocol design.}.
A very important aspect of Design Science Research is \textit{evaluation} of the artifact.
The evaluation is the process that uncovers new knowledge, and separates the process from routine design~\cite[p.~7]{alanhevnerDesignResearchInformation2010}.
There are many aspects that could be evaluated, but the aspects that \textit{should} be evaluated are those that are related to the reason for creating the artifact in the first place; the aspects related to the research objectives~\cite[p.~115]{oatesResearchingInformationSystems2006}.


\subsection{The General Design Cycle }

\paragraph{Design Cycle}
Problem solving by design can follow a general design cycle.
This is a circular and iterative process.
The reasoning that occurs in a design cycle, and the knowledge generated during a cycle, is illustrated in \cref{fig:dsrpm}~\cite[p.~26]{alanhevnerDesignResearchInformation2010}.

\begin{figure}[htbp]  % order of priority: h here, t top, b bottom, p page
  \centering
  \includegraphics[width=\textwidth]{figures/dsrm-flow.pdf}
  \caption[Design Science Research Process Model]{\textbf{Design Science Research Process Model}. The general process followed by Design Science Research. Design begins with awareness of a problem, and progresses through a suggestion for a solution, to development, evaluation and a conclusion.
  The stages produce different outputs, shown in the right column.
  After the conclusion, new knowledge is contributed.
  There is also knowledge produced by development and evaluation, nicknamed ``circumscription''.
  This knowledge is fed back into a new round of suggestion~\cite[p.~11-13]{vijayvaishnaviDesignScienceResearch2019}.
  (Adopted from Figure 3 in \textcite[p.~11]{vijayvaishnaviDesignScienceResearch2019})}\label{fig:dsrpm}
\end{figure}


\paragraph{Awareness of Problem}
The process begins by becoming aware of a problem or opportunities, in the context of humans or an organization.
A proposal for what could be solved is made explicit.


\paragraph{Suggestion}
Then, a suggestion phase begins, where existing knowledge and theories are applied, as well as creativity, to create a tentative design that fits the proposal.
This design could be flawed or incorrect, which is why it is important to realize the design, to detect issues.


\paragraph{Development}
The development phase will build a solution or prototype, aiming to fulfill the suggested design.
This phase will uncover problems, inconsistencies, new learning about the problem, and other related knowledge.
That knowledge is useful for creating a new and improved design.
The quality of the \textit{implementation} of the artifact does not need to be novel, as it is the \textit{design} which is interesting~\cite[p.~12]{vijayvaishnaviDesignScienceResearch2019}.


\paragraph{Evaluation}
After an artifact is created, the evaluation phase will measure the artifact.
The measurements originate from the initial proposal, which holds the criteria for success.
This phase may also discover new knowledge, which can be used later to create a new and improved design~\cite[p.~13]{vijayvaishnaviDesignScienceResearch2019}.


\paragraph{Conclusion}
Finally, the conclusion phase will consolidate the results.
The knowledge gained from the results will either be ``firm'' or ``loose ends''.
\Textcite[p.~13]{vijayvaishnaviDesignScienceResearch2019} describes this as the following:

\begin{quote}
``\textit{Not only are the results of the effort consolidated and ``written up'' at this phase, but the knowledge gained in the effort is frequently categorized as either ``firm'' --- facts that have been learned and can be repeatedly applied or behavior that can be repeatedly invoked --- or as ``loose ends'' --- anomalous behavior that defies explanation and may well serve as the subject of further research.}''
\end{quote}


\subsection{Methodology}

Based on the understanding of design science research, and the steps of a design science research process (\cref{fig:dsrpm}), a six step Design Science Research Methodology has been made by \textcite[p.~28-30]{alanhevnerDesignResearchInformation2010}.
This methodology forms the skeleton of this thesis.
The six steps of the methodology are the following:

\begin{enumerate}
  \item \textit{Problem identification and motivation.}
  \item \textit{Define the objectives for a solution.}
  \item \textit{Design and development.}
  \item \textit{Demonstration.}
  \item \textit{Evaluation.}
  \item \textit{Communication.}
\end{enumerate}


\paragraph{1. Problem identification and motivation}
The specific research problem must be defined.
The definition is used to develop the artifact which solves the problem.
The value of a solution to the problem should be justified as well.
If the value of the solution is justified, it can motivate the researcher and the thesis' audience to pursue the solution and accept the results~\cite[p.~28,29]{alanhevnerDesignResearchInformation2010}.

\paragraph{2. Define the objectives for a solution}
The objectives should be inferred from the problem definition, and the author's knowledge of what is possible and feasible.
The objectives can be how much better a new solution should be (quantitative), or a description of how a new artifact would solve problems that are currently unsolved (qualitative)~\cite[p.~29]{alanhevnerDesignResearchInformation2010}.

\paragraph{3. Design and development}
Create an artifact to solve the problem and fulfill the objectives.
The artifact can be one of the five classes listed in \cref{par:artifact-classes} (constructs, models etc.).
The desired functionality and architecture is determined, and the actual artifact is created~\cite[p.~29]{alanhevnerDesignResearchInformation2010}.


\paragraph{4. Demonstration}\label{par:dsrm-demonstration}
The artifact is demonstrated, to solve instances of the identified problem.
This could be experiments, simulations, case studies etc.~\cite[p.~30]{alanhevnerDesignResearchInformation2010}.


\paragraph{5. Evaluation}
The artifact is observed and evaluated to measure how well it solves the identified problem.
This can be done by comparing the results of the demonstration to the  objectives of an ideal solution.
There are many different ways to evaluate an artifact, and the correct approach should be decided based on the nature of the identified problem.
After evaluation, the researcher can go back to step 3 to improve the design.
If there is not enough time, resources or a need to do so, the process moves to step 6 instead~\cite[p.~30]{alanhevnerDesignResearchInformation2010}.

\paragraph{6. Communication}
The process must be communicated to other researchers and relevant audiences.
This communication includes: the problem and its importance, the artifact and its utility, the rigor of the artifact design, and the effectiveness of the design~\cite[p.~30]{alanhevnerDesignResearchInformation2010}.%
\footnote{This thesis is a central part of this communication.}




\section{Requirements Engineering}

\input{chapters/5-method/2-requirements-engineering.tex}


\section{Development Methodologies}


The case for software development is the same as with requirements engineering: Design Science Research has little guidance.
And again, the software engineering field has the answers.\\

The goal for the development methodology is to \textbf{create the right solution}, which solves the identified problem and fulfils the software requirements.
The methodology also aims to \textbf{avoid or reduce risks} for project failure, by tackling it as early as possible.
Research often deviates from routine design here, by going for the risks first instead of delaying or hiding them, as this may lead to new knowledge~\cite[p.~114]{oatesResearchingInformationSystems2006}.\\

Another goal for the development process is to create ``good'', high quality software, so the project can be accepted by the \gls{open source} developer ecosystem for further development and maintenance.
Bad code or a bad design may result in a full rewrite by the next interested developer, or the developer may try to contribute but find it hard and give up.\\

Development methodology will not follow one strict practice, but rather piece together many different practices and values, which have lead to good results in the author's past.


\subsection{Agile}

Development will follow agile values and principles, as described in \citetitle{kentbeckManifestoAgileSoftware2001}~\cite{kentbeckManifestoAgileSoftware2001} and \citetitle{PrinciplesAgileManifesto}~\cite{PrinciplesAgileManifesto}.
This means readjusting plans, rapid feedback from stakeholders, and software that works underway in development.

\paragraph{Agile development}
As there are many unknown factors in development, such as third party components and services to comply and integrate with, and unknown and hard to use \glspl{API}, the plans and designs may change.
As with software requirements, the data structures, algorithms and design in the software solution will have to change as the developer learns the systems and problem space better.
\textbf{Responding to change} will be valued more than following a plan here.
Also, \textbf{working software} is more valuable than extensive documentation, meaning that code comments, tests, design specifications and diagrams will be given less effort than code, particularly if done up-front before the code.
The alternative is that this documentation is made, but the code for it quickly proves itself impossible to make, or there is not enough time to implement it, leaving only useless documentation as the result.
This also ties in to \textbf{simplicity and maximizing work-not-done}.\\

Regular reflection will be used weekly or bi-weekly, to assess if the process can be more effective.
Sometimes tools and technologies may seem like a good fit for the development, but instead wastes more time than the developer productivity provided.
Retrospective analysis of the development progress will try to detect this, and then expose if bad approaches are used.
If so, these will be removed or replaced if possible.\\

Stakeholder involvement is important as well.
The development will have a stakeholder as the developer (the author), which knows how the artifact will be used.
Additionally, the supervisor will see a demo during development, to provide feedback and help prioritize the next steps.


\subsection{Iterative Development}

The software system will be developed iteratively.
This means the components will be implemented up to a threshold of functionality, and executed to evaluate the behavior.
The evaluation is not a formal and rigorous one, but rather informal and aims to quickly confirm if the software has the correct behavior.
Then the components are developed some more, in a loop until the project ends.\\

The components are also developed incrementally.
It also means a component will be worked on until it reaches \textit{some} functionality, and then the next component will be worked on until it is on par.
No component is developed to completion while the others are not started on.


\subsection{Lean and Minimum Viable Product}

Lean development is a set of principles inspired from Lean Manufacturing (for automobiles and such)~\cite{rachellelynnGuidingPrinciplesLean}.


\paragraph{Eliminate waste}
A core principle is to \textbf{eliminate waste}.
This is also seen in Agile, as maximizing work-not-done.
What Lean regards as waste is any work and output that does not have value.
This means avoiding: unnecessary code and functionality (things deemed ``could be nice to have''), unfinished work in progress (code and features not completed), defects and poor quality (do it well, do it once).


\paragraph{Build quality in}
Another principle is \textbf{building quality in}.
Important business logic should have automated unit tests (however not all code needs tests\footnote{A project with high uncertainty and changing requirements may find tests a hinder as they have to up updated all the time. This results in extra work and rework.})
Tedious and repetitive tasks should be automated, for example with scripts and tools.


\paragraph{Create knowledge}
A third principle is to \textbf{create knowledge}.
Code will have comments where needed\footnote{Not all comments are good comments. Code with side effects, strange design choices or hard to read implementation may need clarification by using comments.}.
Documentation will explain the software at a high level.


\paragraph{Deliver fast}
The last principle applied is to \textbf{deliver fast}.
While there are no students to use the artifact now, the focus is still on creating a functioning solution as early as possible.
This is done by prioritizing the requirements in an order to get a \textit{minimum viable product} (MVP).
This is a solution that is just barely usable.
The goal is to get it into the hands of users as quickly as possible, because this creates valuable feedback.

\subsection{Tracer Bullets}

\paragraph{Tracer bullets}
Using Tracer Bullets in code is a metaphor from \citetitle{huntPragmaticProgrammerJourneyman2000}~\cite{huntPragmaticProgrammerJourneyman2000}.
When using a machine gun, the operator does not precisely calculate where to shoot ahead of time.
Instead, tracer bullets are occasionally loaded into the gun, which glow up and give visual feedback to the operator as to where the bullets travel.


\paragraph{Tracer code}
The same idea applied to coding means that uncertainty and risk is not dealt with by heavy upfront planning.
A software solution likely has to interface with many unknown parts, adding risk for each one.
The developer creates ``tracer code'', which has the goal of quickly connecting all the components and parts of the system, without adding lots of functionality.
Stubs and empty code can be used so the code compiles, as long as all the actual components in the final design are integrated and running.
If any problems are discovered, then adjust and redesign as necessary~\cite{huntPragmaticProgrammerJourneyman2000}.\\

The tracer code is real code, not a prototype.
Therefore, it should be made with proper quality.
It just lacks functionality.
With a working skeleton system, more functionality can be added on top.


\subsection{Domain-Driven Design}\label{sec:ddd}

Domain-driven design is an approach to domain modeling in software.
The goal is to create a domain model in software that uses the same language a domain expert uses, to create better, more evolvable and understandable code.
If the domain has for example ``trees'' and ``nodes'', the code will also use trees and nodes as names, making stakeholder discussion straight forward.
The code will accurately model the domain, increasing understanding and capturing the details of the business logic~\cite{evansDomaindrivenDesignTackling2004}.
The model is represented as normal, executable code.
There is no specific file representation, and no \acrshort{MDD} framework involved.
This is not a \acrshort{MDD} method --- it is a object oriented coding and naming method.\

Another element from Domain-driven design is to use a layered software architecture.
The domain model is isolated from the rest of the code.
This makes it easy to see and reason about the behavior of the code, in terms of business logic~\cite[p.~69]{evansDomaindrivenDesignTackling2004}.
The alternative is that business logic is spread around the system, partially in the user interface components, persistence logic, and so on.
The layers makes every aspect of the program more cohesive and makes interpretation of the designs easier~\cite[p.~69]{evansDomaindrivenDesignTackling2004}.\\

Practically, the domain model will be in its own layer, and the user interface will be in its own layer.
The user interface will depend on the domain model.
The domain model will be unaware of any user interface.
An illustration is shown in \cref{fig:ddd-layered}.

\begin{figure}[htbp]  % order of priority: h here, t top, b bottom, p page
  \centering
  \includegraphics{figures/layered.pdf}
  \caption[Layered Architecture]{\textbf{Layered architecture}. The components above depends on the components below, but not vice versa.}\label{fig:ddd-layered}
\end{figure}


\subsection{Test-Driven Development}

Test-driven development is about creating automated tests before writing the implementation~\cite[p.~105]{knibergScrumXPTrenches2015}.
It will be used sparingly, for cases where the behavior is complex and important to get right.
The developer will have to judge when it is needed, based on expected complexity, requirements and behavior for a unit of code.
This may be especially relevant for some business logic in the domain model.
Writing the tests first also creates a more testable design~\cite[p.~106]{knibergScrumXPTrenches2015}.\\

The benefit of automated tests is confidence in the code against bugs.
It also helps for when other developers join the project, as they can be confident about making changes without breaking existing code.
A goal is to have other contributors develop this project further, and therefore avoiding ``legacy code'' is a good thing.
The author of \citetitle{feathersWorkingEffectivelyLegacy2005} writes:

\begin{quotation}
``Legacy code is somebody else’s code. But in programmer-speak, the term means much more than that. 
[\ldots]


In the industry, \textit{legacy code} is often used as a slang term for difficult-to-change code that we don’t understand. 
[\ldots]


To me, \textit{legacy code} is simply code without tests. [\ldots]


Code without tests is bad code. It doesn’t matter how well written it is; it doesn’t matter how pretty or object-oriented or well-encapsulated it is. With tests, we can change the behavior of our code quickly and verifiably. Without them, we really don’t know if our code is getting better or worse.''


---~\textcite{feathersWorkingEffectivelyLegacy2005}
\end{quotation}


\subsection{Prototyping}

Prototyping will be used to create simple implementations when there is uncertainty of the design and big risks.
A prototype will create learning by coding in the real environment, and then prototype code is discarded afterwards.
A prototype can test feasibility and reveal good and bad sides of a design, quickly and cheaply.\\

The main bulk of prototyping has already been performed, as part of the pre-project in \cite{rekstadModelingEnvironmentCloud2020}.



\section{Evaluation}


This section will describe how the evaluations for the artifacts were made.
The evaluations try to test for value in the solution design, by comparing how well the artifacts solve the identified problem.

\subsection{Software Artifact}

The built artifact is evaluated according to the Design Science Research methodology in \cref{par:dsrm-demonstration}.

\paragraph{Assumptions}
The functionality of the original \gls{Eclipse} editors for \gls{Ecore} is assumed to be correct and useful for students.
The functionality is also required, in order to effectively use \acrshort{EMF} for \acrfull{MDD}.

\paragraph{Demonstration goal}
Therefore, a demonstration should show the presence of the original functionality from \gls{Eclipse} in the new artifact.
To do this, the artifact will be used to complete \textit{use cases}, based on the modeling approach used in \gls{TDT4250} (see \cref{par:tdt4250-methodology}).


Additionally, a goal is to not use the \gls{Eclipse}, and a goal is to perform the use cases in a \gls{cloud} based \gls{IDE}, in this instance \gls{Gitpod}.


\paragraph{Evaluation of demonstration}
The evaluation will be a list of tests with modeling actions from \cref{par:tdt4250-methodology}.
A test is successful if the tester (the author) can perform the action, and without using \gls{Eclipse} and also doing it in \gls{Gitpod}.


\subsection{Open Source Viability}

\paragraph{Evaluation goal}
A goal of this thesis is that the artifact's source code is developed further, by either master students, the Eclipse ecosystem, or other contributors with interest in \acrshort{EMF} or tree editors.
The strategy to solve this is by making the source code \gls{open source}.


Therefore, the source code will be evaluated to indicate how fit it is to be an \gls{open source} project.


\paragraph{Test criteria}
To test how fit the project is, a checklist is synthesized from online guides for open source projects.
The sources are highly reputable, such as the Eclipse Foundation, the GitHub community, and sites endorsed by these.
The criteria will check for presence of elements or properties of the project, and succeed if it is present.
A qualitative evaluation will proceed, to conclude the test results.

