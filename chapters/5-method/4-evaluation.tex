
This section will describe how the evaluations for the artifacts were made.
The evaluations try to test for value in the solution design, by comparing how well the artifacts solve the identified problem.

\subsection{Software Artifact}

The built artifact is evaluated according to the Design Science Research methodology in \cref{par:dsrm-demonstration}.

\paragraph{Assumptions}
The functionality of the original \gls{Eclipse} editors for \gls{Ecore} is assumed to be correct and useful for students.
The functionality is also required, in order to effectively use \acrshort{EMF} for \acrfull{MDD}.

\paragraph{Demonstration goal}
Therefore, a demonstration should show the presence of the original functionality from \gls{Eclipse} in the new artifact.
To do this, the artifact will be used to complete \textit{use cases}, based on the modeling approach used in \gls{TDT4250} (see \cref{par:tdt4250-methodology}).\\


Additionally, a goal is to not use the \gls{Eclipse}, and a goal is to perform the use cases in a \gls{cloud} based \gls{IDE}, in this instance \gls{Gitpod}.


\paragraph{Evaluation of demonstration}
The evaluation%
\footnote{The evaluation is classified as \textit{ex ante} and \textit{artificial}, for those familiar with the approach by \textcite{sonnenbergEvaluationsScienceArtificial2012} and \cite{venableComprehensiveFrameworkEvaluation2012}.} 
will be a list of tests with modeling actions from \cref{par:tdt4250-methodology}.
A test is successful if the tester (the author) can perform the action, and without using \gls{Eclipse} and also doing it in \gls{Gitpod}.


\subsection{Open Source Viability}

\paragraph{Evaluation goal}
A goal of this thesis is that the artifact's source code is developed further, by either master students, the Eclipse ecosystem, or other contributors with interest in \acrshort{EMF} or tree editors.
The strategy to solve this is by making the source code \gls{open source}.\\

Therefore, the source code will be evaluated to indicate how fit it is to be an \gls{open source} project.


\paragraph{Test criteria}
To test how fit the project is, a checklist is synthesized from online guides for open source projects.
The sources are highly reputable, such as the Eclipse Foundation, the GitHub community, and sites endorsed by these.
The criteria will check for presence of elements or properties of the project, and succeed if it is present.
A qualitative evaluation will proceed, to conclude the test results.
