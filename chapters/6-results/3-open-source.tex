
This section will describe the measures taken in order to make the project%
\footnote{The results report on the project in version \texttt{59b722c117}, available at \href{https://github.com/krissrex/tdt4900-master-thesis-ecore-tree-editor/tree/59b722c117346dcc53da16275819e0d5952f0d05}{\nolinkurl{https://github.com/krissrex/tdt4900-master-thesis-ecore-tree-editor/tree/59b722c117346dcc53da16275819e0d5952f0d05}}.}
viable and maintainable as an \gls{open source} project.

\subsection{Code Availability}

Possibly the most important part of \gls{open source}, is available source code.
The project is hosted%
\footnote{Project source: \href{https://github.com/krissrex/tdt4900-master-thesis-ecore-tree-editor}{\nolinkurl{https://github.com/krissrex/tdt4900-master-thesis-ecore-tree-editor}}.}
on a public website for collaboration on \gls{open source} software: \gls{GitHub}.

Also important, is the project visibility being \textit{public}, not private.

The project has the supervisor added as a contributor, in case one project maintainer is unavailable.

\subsection{Documentation}

\paragraph{Readme}
The main project has a ``Readme'' file with an overview of the project's components.

The components named ``tree-document-model-js'', ``tree-editor-frontend'', ``vscode-ecore-tree-editor-extension'' and ``vscode-webview-tree-editor-rpc'' have a Readme.
The ``model-server'' component does not have a Readme.\\

All the readme files are either very minimal, or the default Readme from a project generator.


\paragraph{Source code}
All the modules contain some comments inside the source code.
Not all the source code is documented, only where the author deemed it necessary.
A code base search%
\footnote{
\texttt{
ag --stats -c --ignore-dir dist '\textbackslash{}Q/**\textbackslash{}E\textbackslash{}s' .
}}
\lstinline{ag --stats -c --ignore-dir dist '\Q/**\E\s' .}
returned that 58 files of 169 files had comments, with a total of 128 comments.

\subsection{Automation}

\paragraph{Package manager}
A package manager is used for installing dependencies and compiling each module individually.
For the TypeScript modules, \texttt{npm} (Node Package Manager) is used, and dependencies are tracked in a \texttt{package.json}.
For the java module, \texttt{mvn} (Apache Maven) is used, and dependencies are tracked in a \texttt{pom.xml}.

\paragraph{Build}
Build scripts using \texttt{bash} are provided, that compile the modules (using npm or mvn) and copy the outputs to the correct path.
They also build in the correct order, regarding inter-module dependencies.

\paragraph{IDE configuration}
Files are added to automatically configure a contributor's \acrshort{IDE}, if they use \gls{VSCode} for the TypeScript modules and IntelliJ for the java module.
When using \gls{VSCode}, a list of recommended extensions is provided as well, which can be automatically installed.
There are \textit{Tasks} defined for \gls{VSCode} that can trigger the different npm builds, and \textit{Run configurations} to start the modules.

\paragraph{CI/CD}
There is no Continuous Integration (CI) and Continuous Deployment (CD) configured.
This can be added later when needed; for 1 developer it is overhead.

\subsection{Licensing}

\paragraph{Module license}
The modules use the MIT license\footnote{https://opensource.org/licenses/MIT}.
It is a very simple and permissive licence, compatible with \gls{open source}, and commonly used.
The licenses are not in separate files or the readme.
They are instead mentioned in the \texttt{package.json} and \texttt{pom.xml} files.

\paragraph{Copied code}
Some code is copied from other sources.
The original license has been included in these cases.
No code is copied from incompatible or strict licenses that contradict MIT.

\paragraph{Third party dependencies}
No proprietary dependencies are used%
\footnote{TypeScript modules were scanned with: \lstinline{npx license-checker --production}.}
, and none with incompatible or intrusive licenses.

\subsection{Code}

\paragraph{Code style}
The code uses readable names and small files.
The programming languages (TypeScript and Java) are common, especially in this context.
The code is formatted with automatic code formatters\footnote{Prettier and IntelliJ format the code.}, ensuring a consistent style.

\paragraph{Dependencies}
The dependencies and libraries used are common and in some cases official, in this context.
Effort has been put into using the same dependencies as related works (such as \acrshort{LSP} and EMF.Cloud projects).


\subsection{Issue Tracking}

An issue tracker is available on \gls{GitHub}.
A user is required, but signup is free.
A discussion forum is available as well on \gls{GitHub}.
There is no Wiki, but it is easy to create one on \gls{GitHub} if demand arises.

