A central design result is the constructs used to represent trees in an editor.
These constructs are referred to as the \textbf{domain model} or simply \textbf{model} in this section (not to be confused with an \acrshort{Ecore} model).
This representation is what the frontend presents to a user, and what the extension and server is using to communicate over \acrfull{TLSP}.
Knowing this model is essential for communicating how the design works.
This is because it is used as the ``ubiquitous language'' formed by Domain-driven design (see \cref{sec:ddd} and \cite{evansDomaindrivenDesignTackling2004}).
Note that \textbf{the domain model is generic for all trees and unaware of \acrshort{EMF}}.
This is because it aims to be a domain model for the \acrshort{TLSP}, reuseable for other use cases than \acrshort{EMF} editing.
This section will explain where some of the names come from, and what these data structures look like.\\

\subsection{Borrowed Terms}

\paragraph{Trees}
When words like tree, node, root and children are used, they refer to the concepts for tree structures described in \cref{sec:tree-structures}.
A tree has exactly one root.
The root can have nodes as children, and these can have children again.
A node has a name and an icon that can represent it in a hierarchical tree structure.

\paragraph{Icon}
An icon is a visual representation --- a picture, illustration, symbol --- that represents some information about a node.
An icon can show how one node differs from another, like what type it is, or it can show if a node is invalid or not.

\paragraph{DataUri}
This is a more technical term.
The trees will be displayed on the web, with icons.
A way to store icons as text is using the HTML data-uri scheme.
It is just text, but has a semantic meaning.
When set as the image source in a web browser, it will be displayed as a picture.
A data-uri starts with a prefix which specifies the scheme, the content type and the encoding, for example: \lstinline|data:image/gif;base64,|.
Then, it is followed by the encoded image.

\paragraph{Document}
Because this is for an editor, the concept of documents are borrowed from \gls{VSCode}.
A document is the editor's representation of a file that the user wants to modify.
When a editor window is open in an \acrshort{IDE}, this editor shows a single document.
The document can be opened, modified, saved, renamed and so on.


\subsection{The Domain Model}

The domain model is specified in this thesis by using \gls{TypeScript}, but it can be translated to other languages, such as java.\\

A brief summary of the more exotic features of \gls{TypeScript} may be helpful for the reader:
Note that the `?' means optional or nullable.
\gls{TypeScript} can also ``alias'' types, meaning a new type can be defined by simply renaming an existing type.
This can put more meaning into types like \texttt{string} and \texttt{number}, especially when they are reused in multiple places.
Some builtin interfaces are used, like \texttt{Array} (a list) and \texttt{Record} (an object, or dictionary/map-like structure with keys and values).\\

The domain model will now be presented.
It consists of the following elements: \texttt{TreeDocument}, \texttt{TreeRoot}, \texttt{TreeNode}, \texttt{NodeIcon}, \texttt{IconConfiguration}, \texttt{HierarchyConfiguration}, \texttt{Action}, \texttt{ActionEvent}, \texttt{ActionConfiguration} and \texttt{EditorState}.
It also defines the following aliases for strings: \texttt{ActionId}, \texttt{IconDataUri}, \texttt{NodeId} and \texttt{NodeType}.
These concepts are explained below.

\paragraph{TreeDocument}
The main data structure is the \texttt{TreeDocument}, in \cref{lst:result-tree-document}.
This holds a list of \texttt{TreeRoot}s.
A document can have multiple roots, because there can be related trees.
For example in \acrshort{EMF}, the \texttt{.genmodel} file has a related \texttt{.ecore} file.
Opening the GenModel would also show the \gls{Ecore} model, in an editor with two roots.

\begin{lstlisting}[language=Typescript, label={lst:result-tree-document}, caption={[TreeDocument]TreeDocument TypeScript code.}]
interface TreeDocument {
  roots: Array<TreeRoot>;
}
\end{lstlisting}

\paragraph{TreeRoot}
The \texttt{TreeRoot} in \cref{lst:result-tree-root} holds references to the tree's root node.
It also holds the configurations for how the nodes should be given icons, what actions a user can perform on the nodes, and what a valid node hierarchy looks like.
The actions, hierarchy and icons use the \texttt{type} property of \texttt{TreeNode} to enforce this.
The \texttt{TreeRoot} has an id as well, to separate it from other roots.
This id must be unique inside the TreeDocument.\\

A \texttt{TreeRoot} does not require a root node, for example in the case the \texttt{TreeRoot} was just created or the node was deleted.

\begin{lstlisting}[language=Typescript, label={lst:result-tree-root}, caption={[TreeRoot]TreeRoot TypeScript code.}]
interface TreeRoot {
  id: string;
  rootNode?: TreeNode;
  actions: ActionConfiguration;
  hierarchy: HierarchyConfiguration;
  icons?: IconConfiguration;
}
\end{lstlisting}


\paragraph{TreeNode}
For representing the nodes themselves, there is the \texttt{TreeNode} in \cref{lst:result-tree-node}.
It has an id that is unique in the \texttt{TreeDocument}.
The id is a string, but aliased to \texttt{NodeId} in \gls{TypeScript}.\\

Next, the \texttt{TreeNode} has a \texttt{type}, which is very important.
This type is a string, for example ``EClass'' or ``EAttribute'', and decides the icon, the allowed child nodes, and the possible actions a user can perform on this node.
The string is aliased as \texttt{NodeType} in the model.\\

The \texttt{name} is what shows up in the hierarchical tree structure when presented to the user.
It also reflects a property the user can edit.
The name can for example be ``MyClass'', ``Organization'' or ``NTNU''.\\

To help inform the user what a node represents, the \texttt{documentation} property can hold a help string.
The user interface could show this on hover, or when a node is selected in a designated help area.\\

The \texttt{TreeNode} holds instances of other \texttt{TreeNode}s in the \texttt{children}-property.
This is what enables the tree structure to be represented.\\

Sometimes, a node can be special, for example invalid.
To indicate this, the optional \texttt{iconOverride} can specify a new icon instead of the one from the \texttt{TreeRoot}'s icon configuration.\\

The last property is the \texttt{EditorState}.
This has the properties \texttt{selected} and \texttt{collapsed}, used to hold presentation information about the node.
Being collapsed means that the children are not shown.

\begin{lstlisting}[language=Typescript, label={lst:result-tree-node}, caption={[TreeNode]TreeNode TypeScript code.}]
interface TreeNode {
  id: NodeId;
  type: NodeType;
  name?: string;
  documentation?: string;
  children: Array<TreeNode>;
  iconOverride?: IconDataUri | NodeIcon;
  editorState?: EditorState;
}
\end{lstlisting}


\paragraph{Action}
As mentioned, a user can perform actions.
These can be validating an \gls{Ecore} model, creating a new dynamic instance, and so on.
This is represented by an \texttt{Action} in \cref{lst:results-action}.
The purpose of the \texttt{Action} is to show the user something they can perform, but only contain enough information so it can be sent back to the Tree Language Server to be performed there.
Essentially, an \texttt{Action} is like a reference to a procedure on the server.\\

The action has an \texttt{id}, which is unique in the \texttt{TreeRoot}.
This is what the server uses to know which procedure should be executed.
The \texttt{name} and optional \texttt{icon} are for presenting the \texttt{Action} to the user.

\begin{lstlisting}[language=Typescript, label={lst:results-action}, caption={[Action]Action TypeScript code.}]
interface Action {
  id: ActionId;
  name: string;
  icon?: IconDataUri;
}
\end{lstlisting}

\paragraph{ActionConfiguration}
The list of all \texttt{Action}s live under the \texttt{TreeRoot}, inside the \texttt{ActionConfiguration}.
The intention is that the user is presented with an action bar, or other list of actions, which can change depending on the selected \texttt{TreeNode}.
This \texttt{ActionConfiguration} in \cref{lst:result-action-config} also specifies what actions are always shown in such an action bar, in the \texttt{defaultActionbarActions}.\\

The mapping of \texttt{ActionId}s to lists of \texttt{NodeType}s in \texttt{nodeActions} is used when a node is selected.
Each id can be examined to see if it supports the given \texttt{NodeType}.
The type is supported if it is present in that \texttt{ActionId}'s list.


\begin{lstlisting}[language=Typescript, label={lst:result-action-config}, caption={[ActionConfiguration]ActionConfiguration TypeScript code.}]
interface ActionConfiguration {
  availableActions: Array<Action>;
  defaultActionbarActions?: Array<ActionId>;
  nodeActions?: Record<ActionId, Array<NodeType>>;
}
\end{lstlisting}


\paragraph{ActionEvent}
When the user triggers an action from the frontend, it is sent as an \texttt{ActionEvent} to the server.
The \texttt{ActionEvent} is shown in \cref{lst:result-action-event}.
To reference which \texttt{Action} was triggered, the \texttt{ActionId} is set in the \texttt{action} property.
The server may also want to know what \texttt{TreeRoot} the selected node was in, at the time of triggering the action.
Because actions can operate on specific nodes, like creating a new dynamic instance in \acrshort{EMF}, the currently selected \texttt{TreeNode}'s id are set as the \texttt{targetNodes}.

\begin{lstlisting}[language=Typescript, label={lst:result-action-event}, caption={[ActionEvent]ActionEvent TypeScript code.}]
interface ActionEvent {
  targetNodes?: Array<NodeId>;
  action: ActionId;
  targetRoot: TreeRoot;
}
\end{lstlisting}

\paragraph{NodeIcon and IconDataUri}
A \texttt{TreeNode} and an \texttt{Action} can have icons.
There are also icons in the \texttt{TreeRoot}'s \texttt{IconConfiguration}.
A single image is specified as \texttt{IconDataUri}, which is just an alias to a string type.\\
However, for more complex icons like the nodes', an editor may want to layer or alter the icons.
It could be to add multiplicity information, validity state, or other variants of an icon.
This is supported through composition with the \texttt{NodeIcon}.
It defines a list of \texttt{IconDataUri}s, which are drawn from bottom to top, stacked on each other.

\paragraph{HierarchyConfiguration}
The final element presented is the \texttt{HierarchyConfiguration} in \cref{lst:result-hierarchy-config}.
The \texttt{roots} specify what is allowed to be a \texttt{TreeRoot}'s \texttt{rootNode}.
For example, ``EPackage'' could be such a \texttt{NodeType}.\\

The \texttt{allowedChildren} specifies a mapping between a parent's \texttt{NodeType} and its possible children's \texttt{NodeType}.
This is designed with node creation an drag-and-drop in mind.
A mapping could for example be ``EClass'' to ``EAttribute'', ``EAnnotation'' and ``EReference''\footnote{This mapping is an example, and not complete.}.

\begin{lstlisting}[language=Typescript, label={lst:result-hierarchy-config}, caption={[HierarchyConfiguration]HierarchyConfiguration TypeScript code.}]
interface HierarchyConfiguration {
  roots: Array<NodeType>;
  allowedChildren: Record<NodeType, Array<NodeType>>;
}
\end{lstlisting}