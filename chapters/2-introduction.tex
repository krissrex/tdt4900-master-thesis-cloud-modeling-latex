\chapter{Introduction}\label{chap:introduction}

% TODO

* NTNU, TDT4250, EMF in education, Eclipse IDE for EMF, shift from Eclipse to web with Theia/VSCode

* Eclipse ecosystem is moving to the cloud, EMF.Cloud, GLSP, Theia, gitpod, Sprotty.
  * Pre-project identified that the main actors are Typefox, EclipseSource, RedHat, Obeo
  * Main focus on development, not the developer; code in Eclipse and run in cloud

* The pre-project identified a need for a web-based tree editor to work with EMF. It suggested an architecture and design. The ecosystem is already working on graphical editors for EMF, as presented on EclipseCon2020.
  * Some early identified requirements:
    * EMF is big, we don't want to re-implement it. Focus on re-use of existing software
    * Open Source, to increase the chance of adoption and further improvement. The solution will require more work than a single semester.
    * Special needs for project management, because it must be maintained and improved over a longer period of time.
  * Functional requirements were extracted in pre-project from Eclipse IDE's EMF tree editors.
    * View models of different levels, from Ecore metamodel to model instances.
    * View model as a tree based on containment properties.
    * Create empty model files with the minimum file contents.
    * Edit model hierarchies by creating, deleting or moving tree nodes.
    * Edit tree node properties by using a form-based editor.
    * Saving model changes to xmi files.
    * Validation of models.
    * Generation of code from models.
  * Non-functional requirements from empirical evidence, to increase the chance of project adoption of the Eclipse ecosystem.
    * Flexibility - customize rendering and logic for different models.
    * Configurability - alter or toggle behavior per-project based on config files.
    * Conforming to existing architectures that are empirically validated, and familiar to the Eclipse ecosystem developers.