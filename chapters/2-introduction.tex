\chapter{Introduction}\label{chap:introduction}

\section{Model-Driven Development in Education at NTNU}
% * NTNU, TDT4250, EMF in education, Eclipse IDE for EMF, shift from Eclipse to web with Theia/VSCode

\paragraph{In a world that becomes more digital for each day, there is a large need for software development.} Software is often created by writing code using programming languages that compile down to computer instructions. Developers write the code based on a set of requirements, and change it when the  requirements change.

\paragraph{One alternative approach to software development, is \acrfull{MDD}.}
This approach has the developers create models of their \gls{domain}, and this model drives the rest of the software development. 
The code is usually generated from the model. 
If the software requirements change, the model is updated first, and the code is re-generated.
The model itself is often one or more artifacts in the software project, expressed in a modeling language.
Modeling simplifies the \gls{domain} by using abstraction, and reduces the world down to the entities, relations, procedures (or other abstractions) that are needed to solve the relevant problems.


\paragraph{The \acrshort{MDD} approach is taught at \acrfull{NTNU} in a course named \textit{\gls{TDT4250} Advanced Software Design}.}
% TODO

\paragraph{Eclipse IDE is required to work with EMF modeling.}
% students don't like eclipse. reduces chance of adoption and use of emf afterwards.
% TODO

\paragraph{\acrshort{NTNU} wants to move from Eclipse IDE to \gls{VSCode}  running in a web browser.}
% Avoids installation issues with eclipse and java etc. Makes it easier to publish assignments from github. 
% TODO


%TODO
\section{The Eclipse Ecosystem Wants to Run Software in the Cloud}

* Eclipse ecosystem is moving to the cloud, EMF.Cloud, GLSP, Theia, gitpod, Sprotty.
  * Pre-project identified that the main actors are Typefox, EclipseSource, RedHat, Obeo
  * Main focus on development, not the developer; code in Eclipse and run in cloud

\section{A Pre-project Identified a Need for a Tree Editor}

* The pre-project identified a need for a web-based tree editor to work with EMF. It suggested an architecture and design. The ecosystem is already working on graphical editors for EMF, as presented on EclipseCon2020.
  * Some early identified requirements:
    * EMF is big, we don't want to re-implement it. Focus on re-use of existing software
    * Open Source, to increase the chance of adoption and further improvement. The solution will require more work than a single semester.
    * Special needs for project management, because it must be maintained and improved over a longer period of time.
  * Functional requirements were extracted in pre-project from Eclipse IDE's EMF tree editors.
    * View models of different levels, from Ecore metamodel to model instances.
    * View model as a tree based on containment properties.
    * Create empty model files with the minimum file contents.
    * Edit model hierarchies by creating, deleting or moving tree nodes.
    * Edit tree node properties by using a form-based editor.
    * Saving model changes to xmi files.
    * Validation of models.
    * Generation of code from models.
  * Non-functional requirements from empirical evidence, to increase the chance of project adoption of the Eclipse ecosystem.
    * Flexibility - customize rendering and logic for different models.
    * Configurability - alter or toggle behavior per-project based on config files.
    * Conforming to existing architectures that are empirically validated, and familiar to the Eclipse ecosystem developers.
    * Legal - open source licence. No use of incompatible license (e.g. GPL)
    * Compatibility - run in Theia and VSCode