% Please add the following required packages to your document preamble:
% \usepackage{booktabs}
% \usepackage[table,xcdraw]{xcolor}
% If you use beamer only pass "xcolor=table" option, i.e. \documentclass[xcolor=table]{beamer}
% \usepackage{longtable}
% Note: It may be necessary to compile the document several times to get a multi-page table to line up properly
\begin{longtable}{@{}lp{4cm}ll@{}}
\toprule
\multicolumn{1}{c}{\textbf{ID}} &
  \multicolumn{1}{c}{\textbf{Use Case}} &
  \multicolumn{1}{c}{\textbf{\begin{tabular}[c]{@{}c@{}}Supported?\\ {[}NA/No/Unknown/\\ Yes/YES{]}\end{tabular}}} &
  \multicolumn{1}{c}{\textbf{\begin{tabular}[c]{@{}c@{}}Requires \\ Eclipse IDE\end{tabular}}} \\* \midrule
\endfirsthead
%
\multicolumn{4}{c}%
{{\bfseries Table \thetable\ continued from previous page}} \\
\toprule
\multicolumn{1}{c}{\textbf{ID}} &
  \multicolumn{1}{c}{\textbf{Use Case}} &
  \multicolumn{1}{c}{\textbf{\begin{tabular}[c]{@{}c@{}}Supported?\\ {[}NA/No/Unknown/\\ Yes/YES{]}\end{tabular}}} &
  \multicolumn{1}{c}{\textbf{\begin{tabular}[c]{@{}c@{}}Requires \\ Eclipse IDE\end{tabular}}} \\* \midrule
\endhead
%
\bottomrule
\endfoot
%
\endlastfoot
%
\rowcolor[HTML]{EFEFEF} 
1  & Create new .ecore model file                                 & YES     & No  \\
2  & View Ecore model by opening the .ecore file                  & YES     & No  \\
\rowcolor[HTML]{EFEFEF} 
3  & Create an EPackage, EClass and EAttributes and EReferences   & Yes     & No  \\
4  & Change the properties of the package, class and attributes   & Yes     & No  \\
\rowcolor[HTML]{EFEFEF} 
5  & Create a new dynamic instance file from an EClass            & Yes     & No  \\
6  & Enter dynamic instance data                                  & Yes     & No  \\
\rowcolor[HTML]{EFEFEF} 
7  & Change the .ecore model by adding a EAttribute to the EClass & Yes     & No  \\
8 &
  \begin{tabular}[c]{@{}p{4cm}@{}}Open the dynamic instance, confirm if it is marked as invalid\\ because the new attribute is not filled in.\end{tabular} &
  Unknown &
  Yes \\
\rowcolor[HTML]{EFEFEF} 
9 &
  \begin{tabular}[c]{@{}p{4cm}@{}}Open the .ecore model file, and add a new validation\\ to the EClass as an EAnnotation. Use the java validation kind,\\ not OCL.\end{tabular} &
  Yes &
  No \\
10 & Create a .genmodel file based on the .ecore file.            & Unknown     & No  \\
\rowcolor[HTML]{EFEFEF} 
11 & Generate java project with the model code                    & Unknown & Yes \\
12 & Write a validation in the java code                          & NA       & No  \\
\rowcolor[HTML]{EFEFEF} 
13 & Load the model code into the IDE, to use the validation      & No      & Yes \\
14 & Edit the model and run the custom validation                 & Yes*     & Yes \\
\rowcolor[HTML]{EFEFEF} 
15 & Generate a user interface or editor plugin                   & No      & Yes \\* \bottomrule
\caption{Use Case-evaluation of the Tree Editor Extension design.
Based on modeling in TDT4250. Each row and ID is a step in the modeling process.
The evaluation result is in the ``Supported?'' and ``Requires Eclipse IDE'' columns.
For tasks that are not applicable to the modeling environment, the NA (Not Applicable) is chosen.
The `Yes' in ``Supported?'' indicates that the design should support it, possibly with further development, but not major redesign.
The all-capital `YES' means the developed artifact has demonstrated it. }
\label{tab:use-case-evaluation}\\
\end{longtable}